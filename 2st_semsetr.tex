\documentclass[a4paper]{article}
\usepackage[utf8]{inputenc}
\usepackage[english, russian]{babel}
\usepackage{indentfirst}
\usepackage{amsmath,amsfonts,amssymb,amsthm,mathtools}
\usepackage{svg}
\usepackage{graphicx}
\usepackage{prftree}
\usepackage{multicol}
\usepackage[12pt]{extsizes}
\usepackage{cancel}
\usepackage{titlesec}
\newtheorem{theorem}{Теорема}
\newtheorem*{theorem*}{Теорема}
\newtheorem{lemma}{Лемма}
\newtheorem*{lemma*}{Лемма}
\theoremstyle{definition}
\newtheorem{definition}{Определение}
\newtheorem*{definition*}{Определение}
\newtheorem*{name}{Обозначение}
\newtheorem*{exmp}{Пример}
\newtheorem*{paradoks}{Парадокс}
\newtheorem*{hypo}{Гипотеза}
\newtheorem{proposition}{Предложение}
\newtheorem*{proposition*}{Предложение}
\newtheorem*{comment}{Замечание}
\newtheorem{comment*}{Замечание}
\newtheorem*{upr}{Упражнение}
\newtheorem{upr*}{Упражнение}
\newtheorem{consequence}{Следствие}
\newtheorem*{consequence*}{Следствие}
\newtheorem{property}{Свойство}
\newtheorem*{property*}{Свойство}
\renewcommand{\thesubsubsection}{П.\arabic{subsubsection}}
\renewcommand{\thesubsection}{\S \arabic{subsection}}
%\renewcommand{\thesection}{Глава \arabic{section}}
\titleformat{\section}{\normalfont \Large \bfseries}{Глава \thesection}{2.3ex plus .2ex}{}
\title{ МАТАН 2 Семестр}
\author{Носорев Константин}
\date{2019\\}

\numberwithin{theorem}{subsection}
\numberwithin{lemma}{subsection}
\numberwithin{definition}{subsection}
\numberwithin{comment}{subsection}
\numberwithin{consequence}{subsection}
\numberwithin{property}{subsection}
\begin{document}
\maketitle
\tableofcontents
\section{Ряды}
\subsection{Определение ряда. Основные свойства}
\setcounter{subsubsection}{-1}

\subsubsection{Конечные суммы}
$$a_1+a_2+\dots+a_n = \sum_{k=1}^{n} a_k$$
\begin{itemize}
 \item $\sum_{k=1}^{n}{a_n} + \sum_{k=1}^{n}{b_n} = \sum_{k=1}^{n}{(a_n+b_n)}$
 \item $\lambda \sum_{k=1}^{n}{a_n} = \sum_{k=1}^{n}{(\lambda a_n)}$
\end{itemize}
\subsubsection{Числовые ряды}
\definition{Числовым рядом называется выражение вида $$\sum_{k=1}^{\infty}{a_k} = a_1 + a_2 + \dots + a_n+ \dots $$ где $a_k \in \mathbb{R}$ - общий член последовательности, \\а $S_1 = a_1,  S_2 = a_1 + a_2, S_n = \sum_{k=1}^{n}{a_k}$ - частичные суммы ряда }
\begin{definition}
 Числовой ряд называется \textit{сходящимся}, если сходится последовательность его частичных сумм\\
 $$\lim_{n\rightarrow \infty}{S_n}=S \text{ - сумма ряда}, \sum_{k=1}^{\infty}{a_k} = S \in \mathbb{R} $$
 Если предел бесконечен или не существует, то ряд \textit{расходится}
\end{definition}
\exmp $\sum_{k=0}^{\infty}{q^k} \text{ - геометрическая прогрессия }$\\
$$S_0 = 1, S_1 = 1+q, S_2= 1 + q +q^2, \dots , S_n = \sum_{k=1}^{n}{q^k} = \frac{1-q^{n+1}}{1-q}$$\\
Если $|q| < 1$, то $S_n \xrightarrow{n\rightarrow \infty} \frac{1}{1-q}$ - ряд сходится\\
Если $|q| > 1$, то $S_n \xrightarrow{n\rightarrow \infty} \infty$ - ряд расходится\\
Если $q = 1$, то $S_n \xrightarrow{n\rightarrow \infty} \infty$ - ряд расходится\\
Если $q = -1$, то $S_n = \begin{cases}
  0, & \text{n=2k}   \\
  1, & \text{n=2k+1} \\
 \end{cases}$ - ряд расходится\\
\subsubsection{Основные  свойства}
\begin{theorem}[Критерий Коши]
 $$\sum_{k=1}^{\infty}{a_k} - \text{ - сходится }\Leftrightarrow\ \forall{\varepsilon}> 0\ \exists{N}:\ \forall{m\geq n > N}\  |\sum_{k=n}^{m}{a_k}|<\varepsilon$$
\end{theorem}
\begin{proof}
 Используя критерий Коши для посл-ти частичных сумм
 $$\sum_{k=1}^{\infty}{a_k}\text{ - сходится }\Leftrightarrow S_n = \sum_{k=1}^{n}{a_k} \text{ - сходится }$$ $$\xLeftrightarrow{\text{По кр. Коши}} \forall{\varepsilon}>0 \exists{N}\ :\forall{m\geq n-1 > N}\  |S_m-S_{n-1}|<\varepsilon$$
 $$\forall{\varepsilon}>0 \exists{N}\ :\forall{m\geq n \geq N+1}\ |\sum_{k=n}^{m}{a_k}|<\varepsilon$$
\end{proof}
\exmp $$\sum_{k=1}^{\infty}{\frac{1}{n}} \text{ - расходится }  \sum_{k=1}^{\infty}{\frac{1}{n^2}} \text{ - сходится } $$
\begin{consequence*}
 Если в ряду изменить произвольным образом конечное число слагаемых, то новый ряд сходится, когда сходится исходный, и новый ряд расходится, если исходный расходится
\end{consequence*}
\comment Сходимость ряда независит от поведения конечного числа слагаемых
\begin{theorem}[Необходимый признак сходимости ряда]
 Если $\sum_{k=1}^{\infty}{a_k}$ - сходится, то $\lim_{n\rightarrow\infty}{a_n}=0$
\end{theorem}
\begin{consequence*}
 Если $\lim_{n\rightarrow\infty}{a_n}\ne0$, то ряд расходится
\end{consequence*}
\begin{proof}
 $$a_n = S_n - S_{n-1} \Rightarrow \lim_{n\rightarrow\infty}{a_n} = \lim_{n\rightarrow\infty}{S_n - S_{n-1}} = 0$$
\end{proof}
\begin{theorem}[Арифметические свойства]
 Пусть ряды $\sum_{k=1}^{\infty}{a_k}\ \sum_{k=1}^{\infty}{b_k}$ - сходятся, тогда
 $$\forall{\lambda,\mu} \in \mathbb{R}\ \sum_{k=1}^{\infty}{(\lambda a_k + \mu b_k)} = \lambda \sum_{k=1}^{\infty}{a_k} + \mu \sum_{k=1}^{\infty}{b_k} \text{ - сходится}$$
\end{theorem}
\begin{proof}
 Пусть $S_n^A = \sum_{k=1}^{n}{a_k} \rightarrow S^A$
 $S_n^B = \sum_{k=1}^{n}{b_k} \rightarrow S^B$\\
 Рассмотрим $\sum_{k=1}^{\infty}{(\lambda a_k + \mu b_k)}$,$$ S_n=\sum_{k=1}^{n}{\lambda a_k + \mu b_k} = \lambda \sum_{k=1}^{n}{a_k} + \mu \sum_{k=1}^{n}{b_k} = \lambda S_n^A + \mu S_n^B\Rightarrow$$
 $$\exists{\lim_{n\rightarrow\infty}{S_n}} = \lim_{n\rightarrow\infty}{(\lambda S_n^a + \mu S_n^B)} =\lambda S^A + \mu S^B$$
\end{proof}
\begin{comment}
В частности $\sum_{k=1}^{\infty}{\lambda a_k} = \lambda \sum_{k=1}^{\infty}{a_k}$
\end{comment}
\subsubsection{Неотрицательные числовые ряды}
Рассмотрим $\sum_{k=1}^{\infty}{a_k}, a_k \geq 0, S_n \nearrow$
\begin{theorem}[Критерий сходимости ряда с неотрицательными числами]
 Ряд, члены короторого неотрицательны, сходится $\Leftrightarrow$ посл-ть частичных сумм ограничена
\end{theorem}
\begin{proof}
 \mbox{}\\
 $\Rightarrow$ Ряд сходится $\xRightarrow{\text{По определению}}$ последовательность частичных сумм сходится $\xRightarrow{\text{По свойству сходящейся посл-ти}} \{S_n\}  $ - ограничена \\
 $\Leftarrow$   $\{S_n\}  $ - ограничена и $S_n \nearrow \xRightarrow{\text{По th. Вейерштрасса}}  \{S_n\} $ - сходится $\xRightarrow{\text{По определению}}$ ряд сходится
\end{proof}
\begin{theorem}[Признак сравнения]
 Пусть $$\exists{N > 0}: \forall{n > N}  \sum_{k=n}^{\infty}{a_k}, \sum_{k=n}^{\infty}{b_k}, a_k \geq 0, b_k \geq 0 \text{ и } a_k \leq b_k $$
 \begin{enumerate}
  \item Из сходимости $\sum_{k=1}^{\infty}{b_k} \Rightarrow$ сходимость ряда $\sum_{k=1}^{\infty}{a_k}$
  \item Из расходимости $\sum_{k=1}^{\infty}{a_k} \Rightarrow$ расходимость ряда $\sum_{k=1}^{\infty}{b_k}$
 \end{enumerate}
\end{theorem}
\begin{proof}
 Конченое число членов ряда не влияет на сходимость $\Rightarrow$ будем считать, что $a_k \leq b_k \forall{k \geq 1}$
 \begin{enumerate}
  \item Пусть $S_n^A=\sum_{k=1}^{n}{a_k}, S_n^B=\sum_{k=1}^{n}{b_k} $ \\
        $$\forall{k \geq 1}\ a_k \leq b_k\ \Rightarrow S_n^A \leq S_n^B\ \forall{n} \text{ , если сходится } \sum_{k=1}^{\infty}{b_k}$$
        $$\text{то }S_n^B \nearrow \text{ и сходится к }S^B \text{ при } n \rightarrow \infty \Rightarrow S_n^A \leq S_n^B \leq S^B \Rightarrow S_n^A \nearrow$$
        $$\text{ ограничена сверху } S^B \xRightarrow{\text{по th 4}} \text{ ряд } \sum_{k=1}^{\infty}{a_k} \text{ сходится }$$
  \item (от противного)Пусть $\sum_{k=1}^{\infty}{a_k}$ - расходится, а $\sum_{k=1}^{\infty}{b_k}$ - сходится, тогда по пункту 1 ряд $\sum_{k=1}^{\infty}{a_k}$ - сходится $\Rightarrow \bot$
 \end{enumerate}
\end{proof}
\begin{theorem}[Признак сравнения в предельной форме]
 Пусть $$\sum_{k=1}^{\infty}{a_k},\ \sum_{k=1}^{\infty}{b_k}, a_k \geq 0\ b_k > 0\ \text{ и }\exists{\lim_{n \rightarrow \infty}{\frac{a_n}{b_n}} = c > 0} \text{ - конечное} $$
 тогда ряды сходятся и расходятся одновременно
\end{theorem}
\begin{proof}
 $$\forall{\varepsilon > 0} \exists{N > 0}:\ \forall{n > N}\ |\frac{a_n}{b_n} - c| < \varepsilon $$
 $$-\varepsilon < \frac{a_n}{b_n} - c < \varepsilon$$
 $$-\varepsilon + c< \frac{a_n}{b_n}  < \varepsilon + c$$
 $$(-\varepsilon + c)b_n < a_n  < (\varepsilon + c)b_n$$
 Возьмем $\varepsilon = \frac{c}{2}$
 $$\exists{N_0 > 0}:\ \forall{n > N_0}\ $$
 $$ \frac{c}{2}b_n < a_n < \frac{3c}{2} b_n$$
 \begin{enumerate}
  \item Пусть $\sum_{k=1}^{\infty}{b_k}$ - сходится $\xRightarrow{\text{по сл-вию из th. Коши}} \sum_{k=N_0}^{\infty}{b_k}$ - сходится $\xRightarrow{\text{по th 3}} \sum_{k=N_0}^{\infty}{\frac{3c}{2} b_n}$ - сходится $\xRightarrow{\text{по th 5}} \sum_{k=N_0}^{\infty}{a_n}$ - сходится $ \xRightarrow{\text{по сл-вию из th. Коши}} \sum_{k=1}^{\infty}{a_n} $ - сходится
  \item Пусть $\sum_{k=1}^{\infty}{b_k}$ - расходится $\xRightarrow{\text{по сл-вию из th. Коши}} \sum_{k=N_0}^{\infty}{b_k}$ - расходится $\xRightarrow{\text{по th 3}} \sum_{k=N_0}^{\infty}{\frac{c}{2} b_n}$ - расходится $\xRightarrow{\text{по th 5}} \sum_{k=N_0}^{\infty}{a_n}$ - расходится $ \xRightarrow{\text{по сл-вию из th. Коши}} \sum_{k=1}^{\infty}{a_n} $ - расходится
 \end{enumerate}
\end{proof}
\subsubsection{Телескопический признак. Эталонный ряд $\sum_{k=1}^{\infty}{\frac{1}{n^p}}$}
\begin{theorem}[Телескопический признак]
 Пусть $a_n \searrow , a_n \geq 0$ ряд $\sum_{k=1}^{\infty}{a_n} $ - сходится $\Leftrightarrow$ сходится $\sum_{k=0}^{\infty}{2^n a_{2^n}}$
\end{theorem}
\begin{proof}
 Правый ряд $a_1 + 2a_2 + 4a_4+\dots$
 $$\text{Рассмотрим }a_2\leq a_2 \leq a_1$$
 $$2a_4\leq a_3 + a_4 \leq 2a_2 $$
 $$ 4a_8 \leq a_5 + a_6 + a_7 +a_8 \leq 4a_4 $$
 $$ \dots$$
 $$ 2^n a_{2^{n+1}} \leq \sum_{k=2^{n}+1}^{2^{n+1}}{a_k} \leq 2^n a_{2^n} $$
 Сложим выражения в левой и правой частях
 $$ a_2 + 2a_4 + 4a_8 + \dots + 2^na_{2^{n+1}} \leq S_{2^{n+1}}^A - a_1 \leq S_n^B$$
 $$\frac{1}{2} (S_{n+1}^B - a_1) \leq S_{2^{n+1}}^A - a_1 \leq S_n^B$$
 Рассмотрим отдельно каждое неравенство
 \begin{enumerate}
  \item $S_{2^{n+1}}^A - a_1 \leq S_n^B$
        Если ряд $S_n^B$ - сходится $\Rightarrow S_{2^{n+1}}^A - a_1$ - сходится $\Rightarrow S_n^A \leq S^B$ и $\{ S_n^A \}\nearrow \xRightarrow{\text{По th. Вейерштрасса}}$ ряд $\sum_{k=1}^{\infty}{a_n}$ - сходится
  \item $\frac{1}{2} (S_{n+1}^B - a_1) \leq S_{2^{n+1}}^A - a_1$
        $$S_{n+1}^B \leq 2S_{2^{n+1}}^A - a_1$$
        Если ряд $\sum_{k=1}^{\infty}{a_n}$ - сходится $\Rightarrow S_n^B \leq 2S_{2^{n+1}}^A - a_1 $ и $\{ S_n^B \}\nearrow\xRightarrow{\text{По th. Вейерштрасса}}$ ряд $\sum_{k=1}^{\infty}{2^n a_{2^n}}$ - сходится
 \end{enumerate}
 \textit{Примечание:} Расхождение доказывается по признаку сравнения\\
\end{proof}
\begin{theorem}
 Ряд $\sum_{k=1}^{\infty}{\frac{1}{n^p}}$ $\begin{cases}
   \text{cходится} ,   & \text{если} p > 1    \\
   \text{расходится} , & \text{если} p \leq 1 \\
  \end{cases}$
\end{theorem}
\begin{proof} \mbox{}\\
 \begin{itemize}
  \item Пусть $p > 1 \Rightarrow 0 < \{\frac{1}{n^p} \} \searrow_0 $
        Рассмотрим ряд из th 7 $\sum_{k=0}^{\infty}{2^n \frac{1}{2^{np}}} = \sum_{k=0}^{\infty}{(2^{1-p})^n}$ - геометрическая прогрессия $q = 2^{1-p} \Rightarrow \sum_{k=1}^{\infty}{(2^{1-p})^n}$ - сходится $\xRightarrow{\text{по th 7}} \sum_{k=1}^{\infty}{\frac{1}{n^p}}$ - сходится
  \item Пусть $p \leq 1$
        $$\frac{1}{n^p} > \frac{1}{n}\ \forall{n} \in \mathbb{N} \Rightarrow \text{ т.к } \frac{1}{n} \text{ - расходится }
         \xRightarrow{\text{по признаку сравнения}} \frac{1}{n^p} \text{ - расходится} $$
 \end{itemize}
\end{proof}
\subsubsection{Признак Коши. Признак Даламбера}
\begin{theorem}[признак Даламбера]
 Пусть $a_n > 0 $ и $\exists{\overline{\lim}_{n \rightarrow \infty}{\frac{a_{n+1}}{a_n}}} = q$, тогда
 \begin{enumerate}
  \item Если $ 0 \leq q < 1$, то ряд $\sum_{k=1}^{\infty}{a_n} $ - сходится
  \item Если $ q > 1$, то ряд $\sum_{k=1}^{\infty}{a_n} $ - расходится
  \item Если $ q = 1$, то признак не работает
 \end{enumerate}
\end{theorem}
\begin{comment}
Признак удобно применять, если в $a_n$ есть $2^n, n!, n^k$
\end{comment}
\begin{proof}
 $$\forall{\varepsilon} > 0\ \exists{N > 0}:\ \forall{n > N}\  |\frac{a_{n+1}}{a_n} - q| < \varepsilon$$
 $$- \varepsilon < \frac{a_{n+1}}{a_n} -q <\varepsilon  $$
 $$ (q-\varepsilon)a_n < \frac{a_{n+1}}{a_n} < (q+\varepsilon)a_n\ \ \ \forall{n} = N + 1, N+2,\dots$$
 \begin{enumerate}
  \item Пусть $ q < 1$ \\
        Возьмем $\varepsilon: \widetilde{q} = q + \varepsilon < 1$ (Например $\varepsilon = \frac{1-q}{2}$)
        Тогда $\forall{n} > N$
        $$ a_{n+1} < a_n \widetilde{q}$$
        $$ N+1: a_{n+2} < a_{n+1} \widetilde{q}$$
        $$ N+2: a_{n+3} < a_{n+2} \widetilde{q} <  a_{n+1} \widetilde{q}$$
        $$ \dots $$
        $$ N+k-1: a_{n+k} < a_{n+k-1} \widetilde{q} < \dots <a_{n+1} \widetilde{q}^{k-1}$$
        Т.к $\sum_{k=1}^{\infty}{\widetilde{q}^k}$ - геометрическая прогрессия ($\widetilde{q}<1$)$\Rightarrow$ сходится $\xRightarrow{\text{по признаку сравнения}} \sum_{n=1}^{\infty}{a_{n+k}}$ - сходится $\xRightarrow{\text{Следствие критерия Коши}} \sum_{n=1}^{\infty}{a_n}$ - сходится
  \item Пусть $q > 1$ \\
        Возьмем $\varepsilon: \widetilde{q} = q - \varepsilon > 1$ \\
        Тогда $\forall{n} > N\ a_n \widetilde{q} <a_{n+1} \xRightarrow{\text{по 1 пункту}} a_{n+k} >a_{n+1} \widetilde{q}^{k-1}$
        $$a_{n+1}\widetilde{q}^{k-1} \xcancel{\xrightarrow{n \rightarrow \infty}}\ \ 0 \Rightarrow a_{n+k}\xcancel{\xrightarrow{n \rightarrow \infty}}\ \ 0 \xRightarrow{\text{по необходимому признаку}} \text{ряд расходится}$$
  \item Пусть $q=1$$$\sum_{n=1}^{\infty}{\frac{1}{n}}:\ \lim_{n \rightarrow \infty}{\frac{\frac{1}{n+1}}{\frac{1}{n}}} = 1   \text{ ряд расходится} $$
        $$\sum_{n=1}^{\infty}{\frac{1}{n^2}}:\ \lim_{n \rightarrow \infty}{\frac{\frac{1}{(n+1)^2}}{\frac{1}{n^2}}} = 1  \text{ ряд сходится}$$
 \end{enumerate}
\end{proof}
\begin{exmp}
 $$\sum_{n=1}^{\infty}{\frac{n^2}{2^n}}, \lim_{n \rightarrow \infty}{\frac{\frac{(n+1)^2}{2^{n+1}}}{\frac{n^2}{2^n}}} = \lim_{n \rightarrow \infty}{\frac{(n+1)^2}{2n^2}} = \frac{1}{2} < 1 \Rightarrow \text{ряд сходится} $$
\end{exmp}
\begin{theorem}[радикальный признак Коши] \mbox{}\\
 Пусть $a_n \geq 0 $ и $\exists{\overline{\lim}_{n \rightarrow \infty}{\sqrt[n]{a_n}}} = q$, тогда
 \begin{enumerate}
  \item Если $ 0 \leq q < 1$, то ряд $\sum_{k=1}^{\infty}{a_n} $ - сходится
  \item Если $ q > 1$, то ряд $\sum_{k=1}^{\infty}{a_n} $ - расходится
  \item Если $ q = 1$, то признак не работает
 \end{enumerate}
\end{theorem}
\begin{comment}
Признак удобно применять, если в $a_n$ есть $2^n, n^k$
\end{comment}
\begin{proof}
 $$\forall{\varepsilon} > 0\ \exists{N > 0}:\ \forall{n > N}\  |\sqrt[n]{a_n}} - q| < \varepsilon$$
 $$- \varepsilon < \sqrt[n]{a_n}} -q <\varepsilon  $$
 $$ (q-\varepsilon) < \sqrt[n]{a_n}} < (q+\varepsilon)\ \ \ \forall{n} = N + 1, N+2,\dots$$
 \begin{enumerate}
  \item Пусть $q < 1 \Rightarrow \exists{\widetilde{q}} = \frac{q+1}{2} | q < \widetilde{q} < 1 \exists{N > 0}:\ \forall{n > N}\ $\\
        $$ \sqrt[n]{a_n}} < \widetilde{q} \Leftrightarrow a_n < \widetilde{q}^n\ \ \ \sum_{n=1}^{\infty}{\widetilde{q}^n} \text{ - геометрическая прогрессия} (\widetilde{q} < 1)} $$
        $$\Rightarrow \text{ ряд сходится} \xRightarrow{\text{по признаку сравнения}} \text{сходится исходный ряд }$$
  \item Пусть $q > 1 \Rightarrow \exists{\{ a_{n_k}\}}: \sqrt[n]{a_{n_k}}} > 1 \Rightarrow a_{n_k} > 1 \Rightarrow \lim_{n \rightarrow \infty}{a_{n_k}} \ne 0 \Rightarrow$ ряд расходится
  \item Пусть $ q = 1$
        $$ \sum_{n=1}^{\infty}{\frac{1}{n} : \lim_{n \rightarrow \infty}{\frac{1}{\sqrt[n]{n}}}} = 1  \text{ ряд расходится} $$
        $$ \sum_{n=1}^{\infty}{\frac{1}{n^2} : \lim_{n \rightarrow \infty}{\frac{1}{\sqrt[n]{n^2}}}} = 1  \text{ ряд сходится} $$
 \end{enumerate}
\end{proof}
\begin{exmp}
 $$\sum_{n=1}^{\infty}{(\frac{n-1}{n})^{n^2}}, \lim_{n \rightarrow \infty}{(\frac{n-1}{n})^{n}}=e^{-1} < 1 \Rightarrow \text{ряд сходится}$$
\end{exmp}
\begin{theorem}[Признак Раабе]
 Пусть $a_n > 0 $ и $\exists{\lim_{n \rightarrow \infty}{n(\frac{a_n}{a_{n+1}}-1)}} = q$, тогда
 \begin{enumerate}
  \item Если $ q < 1$, то ряд $\sum_{k=1}^{\infty}{a_n} $ - расходится
  \item Если $ q > 1$, то ряд $\sum_{k=1}^{\infty}{a_n} $ - сходится
  \item Если $ q = 1$, то признак не работает
 \end{enumerate}
\end{theorem}
\begin{exmp}
 $$\sum_{n=1}^{\infty}{\frac{1 \cdot 3 \cdot 5 \cdot \textit{\dots} \cdot (2n-1)}{2^n \cdot n!}}$$
 \textit{Доломбер: } $$  \lim_{n \rightarrow \infty}{\frac{(1 \cdot 3 \cdot 5 \cdot \textit{\dots} \cdot (2n+1)) 2^n \cdot n!} {2^{n+1} \cdot (n+1)!(1 \cdot 3 \cdot 5 \cdot \textit{\dots} \cdot (2n-1))}} =$$
 $$\lim_{n \rightarrow \infty}{\frac{2n+1}{2\cdot(n+1)}} = 1 $$
 \textit{Раабе: } $$ \lim_{n \rightarrow \infty}{n(\frac{2(n+1)}{2n+1}-1)} = \lim_{n \rightarrow \infty}{\frac{n}{2n+1}} = \frac{1}{2} < 1 \Rightarrow \text{ ряд расходится}$$
\end{exmp}

\subsubsection{Число e, как сумма ряда}

$$e = \lim_{n \rightarrow \infty}{(1+\frac{1}{n})^n} = \lim_{n \rightarrow \infty}{e_n} $$
\begin{enumerate}
 \item $$ e_n = (1 + \frac{1}{n})^n = \sum_{n=0}^{\infty}{C_k^n \frac{1}{n^k}} = \sum_{k=0}^{n}{\frac{n!}{k!(n-k)!} \frac{1}{n^k}} = $$
       $$= \sum_{k=0}^{n}{\frac{(n-k+1)(n-k+2)\dots n}{k! n^k}} < \sum_{k=0}^{n}{\frac{n^k}{k!n^k}}=\sum_{k=0}^{n}{\frac{1}{k!}} = S_n \Rightarrow e_n < S_n $$
 \item Пусть $m<n$
       $$e_n = \sum_{k=0}^{n}{C_n^k \frac{1}{n^k}} > \sum_{k=0}^{m}{C_n^k \frac{1}{n^k}}= $$
       $$ = 1 + 1 + \frac{n(n-1)}{2! n^2} + \frac{n(n-1)(n-2)}{3! n^3}+ \dots + \frac{n(n-1)(n-2)\dots(n-m+1)}{m! n^m} = A_{n,m}$$
       $$ \Rightarrow e_n > A_{n,m} \text{ Зафиксируем m. Тогда при } n\rightarrow \infty \ e \geq S_m$$
 \item $| \Rightarrow e_n < S_n \leq e \text{ и } \{S_n\} \nearrow \xRightarrow{\text{По th Вейерштрасса}} \text{Ряд сходится}$
       $$e = 1 + 1 + \frac{1}{2!} + \frac{1}{3!} + \dots + \frac{1}{n!} + \dots $$
 \item Оценка погрешности при приблежении числа e частичными суммами:
       $$ 0 < e - S_n = \sum_{k=n+1}^{\infty}{\frac{1}{k!}} = \frac{1}{(n+1)!} +\frac{1}{(n+2)!} + \frac{1}{(n+3)!} + \dots =$$
       $$= \frac{1}{(n+1)!} [ 1 + \frac{1}{(n+2)} + \frac{1}{(n+2)(n+3)}+\dots ]<$$$$< \frac{1}{(n+1)!} [ 1 + \frac{1}{(n+1)} + \frac{1}{(n+1)^2} + \frac{1}{(n+1)^3} +\dots] = \frac{1}{(n+1)!} \frac{n+1}{n} = \frac{1}{n \cdot n!}$$
\end{enumerate}
\upr Доказать, что e - иррационально
\upr Доказать, что 2<e<3
\subsection{Признак Лейбница. Признак Дирихле. Признак Абеля}
\subsubsection{}
%\setcounter{theorem}{0}
\definition Ряд вида $a_1 - a_2 + a_3 - a_4 + ... = \sum_{n=1}^{\infty}{(-1)^n-1 a_n}$ называется знакочередующимся, где $\forall{n}\ a_n > 0$
\begin{theorem}[Признак сходимости Лейбница знакочередующихся рядов]
 Пусть дан знакочередующийся ряд $\sum_{n=1}^{\infty}{(-1)^n-1 a_n}\ a_n > 0$\\
 Если $\lim_{n \rightarrow \infty}{a_n} = 0 $ и $\{a_n\}\searrow$, то ряд сходится
\end{theorem}
\begin{proof}
 $$S_{2m} = a_1 - a_2 + a_3 - a_4 + \dots - a_{2m-2} + a_{2m-1} - a_{2m} =$$
 $$= a_1 - (a_2 - a_3) - (a_4 - a_5) - \dots - (a_{2m-2} - a_{2m-1}) - a_{2m} \leq a_1 \Rightarrow $$
 $$ \Rightarrow \{ S_{2m}\} \text{ - ограничена сверху}$$
 $$ S_{2m+2} = S_{2m} + (a_{2m+1} - a_{2m+2}) \geq S_{2m} \Rightarrow \{S_{2m}\}\nearrow |\xRightarrow{\text{по th Вейерштрасса}} \exists{\lim_{n \rightarrow \infty}{S_{2m}}} = S $$
 $$ S_{2m+1} = S_m +a_{2m+1} \xRightarrow{S_m \rightarrow S, a_{2m+1} \rightarrow 0 \text{ При } n\rightarrow \infty} \lim_{n \rightarrow \infty}{S_{2m+1}} = S $$
\end{proof}
\comment Признак достаточный, но не необходимый!
\end{document}
