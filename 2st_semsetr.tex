\documentclass[a4paper]{article}
\usepackage[utf8]{inputenc}
\usepackage[english, russian]{babel}
\usepackage{indentfirst}
\usepackage{amsmath,amsfonts,amssymb,amsthm,mathtools}
\usepackage{svg}
\usepackage{graphicx}
\usepackage{prftree}
\usepackage{multicol}
\usepackage[12pt]{extsizes}
\usepackage{cancel}
\usepackage{titlesec}
\usepackage{hyperref}
\usepackage[framemethod=TikZ]{mdframed}
\newcounter{memories}[subsection]\setcounter{memories}{0}
\renewcommand{\thetheo}{\arabic{subsection}.\arabic{memories}}
\newenvironment{memories}[2][]{%
\refstepcounter{memories}%
\ifstrempty{#1}%
{\mdfsetup{%
frametitle={%
\tikz[baseline=(current bounding box.east),outer sep=0pt]
\node[anchor=east,rectangle,fill=blue!20]
{\strut Воспоминание~\thetheo};}}
}%
{\mdfsetup{%
frametitle={%
\tikz[baseline=(current bounding box.east),outer sep=0pt]
\node[anchor=east,rectangle,fill=blue!20]
{\strut Воспоминание~\thetheo:~#1};}}%
}%
\mdfsetup{innertopmargin=10pt,linecolor=blue!20,%
linewidth=2pt,topline=true,%
frametitleaboveskip=\dimexpr-\ht\strutbox\relax
}
\begin{mdframed}[]\relax%
\label{#2}}{\end{mdframed}}
\graphicspath{{pictures/}}
\DeclareGraphicsExtensions{.pdf,.png,.jpg}
\newtheorem{theorem}{Теорема}
\newtheorem*{theorem*}{Теорема}
\newtheorem{lemma}{Лемма}
\newtheorem*{lemma*}{Лемма}
\theoremstyle{definition}
\newtheorem{definition}{Определение}
\newtheorem*{definition*}{Определение}
\newtheorem*{name}{Обозначение}
\newtheorem*{exmp}{Пример}
\newtheorem*{paradoks}{Парадокс}
\newtheorem*{hypo}{Гипотеза}
\newtheorem{proposition}{Предложение}
\newtheorem*{proposition*}{Предложение}
\newtheorem*{comment}{Замечание}
\newtheorem{comment*}{Замечание}
\newtheorem*{upr}{Упражнение}
\newtheorem{upr*}{Упражнение}
\newtheorem{consequence}{Следствие}
\newtheorem*{consequence*}{Следствие}
\newtheorem{property}{Свойство}
\newtheorem*{property*}{Свойство}
\renewcommand\qedsymbol{$\blacksquare$}
\renewcommand{\thesubsubsection}{П.\arabic{subsubsection}}
\renewcommand{\thesubsection}{\S \arabic{subsection}}
%\renewcommand{\thesection}{Глава \arabic{section}}
\titleformat{\section}{\normalfont \Large \bfseries}{Глава \thesection}{2.3ex plus .2ex}{}
\title{ МАТАН 2 Семестр}
\author{Носорев Константин}
\date{2019\\}
\hypersetup{
       colorlinks=true, %set true if you want colored links
       linktoc=all,     %set to all if you want both sections and subsections linked
       linkcolor=blue,  %choose some color if you want links to stand out
}
\numberwithin{theorem}{subsection}
\numberwithin{lemma}{subsection}
\numberwithin{definition}{subsection}
\numberwithin{comment*}{subsection}
\numberwithin{consequence}{subsection}
\numberwithin{property}{subsection}
\begin{document}
\maketitle
\tableofcontents
\section{Ряды}
\subsection{Определение ряда. Основные свойства}
\setcounter{subsubsection}{-1}

\subsubsection{Конечные суммы}
$$a_1+a_2+\dots+a_n = \sum_{k=1}^{n} a_k$$
\begin{itemize}
 \item $\sum_{k=1}^{n}{a_n} + \sum_{k=1}^{n}{b_n} = \sum_{k=1}^{n}{(a_n+b_n)}$
 \item $\lambda \sum_{k=1}^{n}{a_n} = \sum_{k=1}^{n}{(\lambda a_n)}$
\end{itemize}
\subsubsection{Числовые ряды}
\definition{Числовым рядом называется выражение вида $$\sum_{k=1}^{\infty}{a_k} = a_1 + a_2 + \dots + a_n+ \dots $$ где $a_k \in \mathbb{R}$ - общий член последовательности, \\а $S_1 = a_1,  S_2 = a_1 + a_2, S_n = \sum_{k=1}^{n}{a_k}$ - частичные суммы ряда }
\begin{definition}
 Числовой ряд называется \textit{сходящимся}, если сходится последовательность его частичных сумм\\
 $$\lim_{n\rightarrow \infty}{S_n}=S \text{ - сумма ряда}, \sum_{k=1}^{\infty}{a_k} = S \in \mathbb{R} $$
 Если предел бесконечен или не существует, то ряд \textit{расходится}
\end{definition}
\exmp $\sum_{k=0}^{\infty}{q^k} \text{ - геометрическая прогрессия }$\\
$$S_0 = 1, S_1 = 1+q, S_2= 1 + q +q^2, \dots , S_n = \sum_{k=1}^{n}{q^k} = \frac{1-q^{n+1}}{1-q}$$\\
Если $|q| < 1$, то $S_n \xrightarrow{n\rightarrow \infty} \frac{1}{1-q}$ - ряд сходится\\
Если $|q| > 1$, то $S_n \xrightarrow{n\rightarrow \infty} \infty$ - ряд расходится\\
Если $q = 1$, то $S_n \xrightarrow{n\rightarrow \infty} \infty$ - ряд расходится\\
Если $q = -1$, то $S_n = \begin{cases}
  0, & \text{n=2k}   \\
  1, & \text{n=2k+1} \\
 \end{cases}$ - ряд расходится\\
\subsubsection{Основные  свойства}
\begin{theorem}[Критерий Коши]
 $$\sum_{k=1}^{\infty}{a_k} - \text{ - сходится }\Leftrightarrow\ \forall{\varepsilon}> 0\ \exists{N}:\ \forall{m\geq n > N}\  |\sum_{k=n}^{m}{a_k}|<\varepsilon$$
\end{theorem}
\begin{proof}
 Используя критерий Коши для посл-ти частичных сумм
 $$\sum_{k=1}^{\infty}{a_k}\text{ - сходится }\Leftrightarrow S_n = \sum_{k=1}^{n}{a_k} \text{ - сходится }$$ $$\xLeftrightarrow{\text{По кр. Коши}} \forall{\varepsilon}>0 \exists{N}\ :\forall{m\geq n-1 > N}\  |S_m-S_{n-1}|<\varepsilon$$
 $$\forall{\varepsilon}>0 \exists{N}\ :\forall{m\geq n \geq N+1}\ |\sum_{k=n}^{m}{a_k}|<\varepsilon$$
\end{proof}
\exmp $$\sum_{k=1}^{\infty}{\frac{1}{n}} \text{ - расходится }  \sum_{k=1}^{\infty}{\frac{1}{n^2}} \text{ - сходится } $$
\begin{consequence*}
 Если в ряду изменить произвольным образом конечное число слагаемых, то новый ряд сходится, когда сходится исходный, и новый ряд расходится, если исходный расходится
\end{consequence*}
\comment Сходимость ряда независит от поведения конечного числа слагаемых
\begin{theorem}[Необходимый признак сходимости ряда]
 Если $\sum_{k=1}^{\infty}{a_k}$ - сходится, то $\lim_{n\rightarrow\infty}{a_n}=0$
\end{theorem}
\begin{consequence*}
 Если $\lim_{n\rightarrow\infty}{a_n}\ne0$, то ряд расходится
\end{consequence*}
\begin{proof}
 $$a_n = S_n - S_{n-1} \Rightarrow \lim_{n\rightarrow\infty}{a_n} = \lim_{n\rightarrow\infty}{S_n - S_{n-1}} = 0$$
\end{proof}
\begin{theorem}[Арифметические свойства]
 Пусть ряды $\sum_{k=1}^{\infty}{a_k}\ \sum_{k=1}^{\infty}{b_k}$ - сходятся, тогда
 $$\forall{\lambda,\mu} \in \mathbb{R}\ \sum_{k=1}^{\infty}{(\lambda a_k + \mu b_k)} = \lambda \sum_{k=1}^{\infty}{a_k} + \mu \sum_{k=1}^{\infty}{b_k} \text{ - сходится}$$
\end{theorem}
\begin{proof}
 Пусть $S_n^A = \sum_{k=1}^{n}{a_k} \rightarrow S^A$
 $S_n^B = \sum_{k=1}^{n}{b_k} \rightarrow S^B$\\
 Рассмотрим $\sum_{k=1}^{\infty}{(\lambda a_k + \mu b_k)}$,$$ S_n=\sum_{k=1}^{n}{\lambda a_k + \mu b_k} = \lambda \sum_{k=1}^{n}{a_k} + \mu \sum_{k=1}^{n}{b_k} = \lambda S_n^A + \mu S_n^B\Rightarrow$$
 $$\exists{\lim_{n\rightarrow\infty}{S_n}} = \lim_{n\rightarrow\infty}{(\lambda S_n^a + \mu S_n^B)} =\lambda S^A + \mu S^B$$
\end{proof}
\begin{comment}
В частности $\sum_{k=1}^{\infty}{\lambda a_k} = \lambda \sum_{k=1}^{\infty}{a_k}$
\end{comment}
\subsubsection{Неотрицательные числовые ряды}
Рассмотрим $\sum_{k=1}^{\infty}{a_k}, a_k \geq 0, S_n \nearrow$
\begin{theorem}[Критерий сходимости ряда с неотрицательными числами]
 Ряд, члены короторого неотрицательны, сходится $\Leftrightarrow$ посл-ть частичных сумм ограничена
\end{theorem}
\begin{proof}
 \mbox{}\\
 $\Rightarrow$ Ряд сходится $\xRightarrow{\text{По определению}}$ последовательность частичных сумм сходится $\xRightarrow{\text{По свойству сходящейся посл-ти}} \{S_n\}  $ - ограничена \\
 $\Leftarrow$   $\{S_n\}  $ - ограничена и $S_n \nearrow \xRightarrow{\text{По th. Вейерштрасса}}  \{S_n\} $ - сходится $\xRightarrow{\text{По определению}}$ ряд сходится
\end{proof}
\begin{theorem}[Признак сравнения]
 Пусть $$\exists{N > 0}: \forall{n > N}  \sum_{k=n}^{\infty}{a_k}, \sum_{k=n}^{\infty}{b_k}, a_k \geq 0, b_k \geq 0 \text{ и } a_k \leq b_k $$
 \begin{enumerate}
  \item Из сходимости $\sum_{k=1}^{\infty}{b_k} \Rightarrow$ сходимость ряда $\sum_{k=1}^{\infty}{a_k}$
  \item Из расходимости $\sum_{k=1}^{\infty}{a_k} \Rightarrow$ расходимость ряда $\sum_{k=1}^{\infty}{b_k}$
 \end{enumerate}
\end{theorem}
\begin{proof}
 Конченое число членов ряда не влияет на сходимость $\Rightarrow$ будем считать, что $a_k \leq b_k \forall{k \geq 1}$
 \begin{enumerate}
  \item Пусть $S_n^A=\sum_{k=1}^{n}{a_k}, S_n^B=\sum_{k=1}^{n}{b_k} $ \\
        $$\forall{k \geq 1}\ a_k \leq b_k\ \Rightarrow S_n^A \leq S_n^B\ \forall{n} \text{ , если сходится } \sum_{k=1}^{\infty}{b_k}$$
        $$\text{то }S_n^B \nearrow \text{ и сходится к }S^B \text{ при } n \rightarrow \infty \Rightarrow S_n^A \leq S_n^B \leq S^B \Rightarrow S_n^A \nearrow$$
        $$\text{ ограничена сверху } S^B \xRightarrow{\text{по th 4}} \text{ ряд } \sum_{k=1}^{\infty}{a_k} \text{ сходится }$$
  \item (от противного)Пусть $\sum_{k=1}^{\infty}{a_k}$ - расходится, а $\sum_{k=1}^{\infty}{b_k}$ - сходится, тогда по пункту 1 ряд $\sum_{k=1}^{\infty}{a_k}$ - сходится $\Rightarrow \bot$
 \end{enumerate}
\end{proof}
\begin{theorem}[Признак сравнения в предельной форме]
 Пусть $$\sum_{k=1}^{\infty}{a_k},\ \sum_{k=1}^{\infty}{b_k}, a_k \geq 0\ b_k > 0\ \text{ и }\exists{\lim_{n \rightarrow \infty}{\frac{a_n}{b_n}} = c > 0} \text{ - конечное} $$
 тогда ряды сходятся и расходятся одновременно
\end{theorem}
\begin{proof}
 $$\forall{\varepsilon > 0} \exists{N > 0}:\ \forall{n > N}\ |\frac{a_n}{b_n} - c| < \varepsilon $$
 $$-\varepsilon < \frac{a_n}{b_n} - c < \varepsilon$$
 $$-\varepsilon + c< \frac{a_n}{b_n}  < \varepsilon + c$$
 $$(-\varepsilon + c)b_n < a_n  < (\varepsilon + c)b_n$$
 Возьмем $\varepsilon = \frac{c}{2}$
 $$\exists{N_0 > 0}:\ \forall{n > N_0}\ $$
 $$ \frac{c}{2}b_n < a_n < \frac{3c}{2} b_n$$
 \begin{enumerate}
  \item Пусть $\sum_{k=1}^{\infty}{b_k}$ - сходится $\xRightarrow{\text{по сл-вию из th. Коши}} \sum_{k=N_0}^{\infty}{b_k}$ - сходится $\xRightarrow{\text{по th 3}} \sum_{k=N_0}^{\infty}{\frac{3c}{2} b_n}$ - сходится $\xRightarrow{\text{по th 5}} \sum_{k=N_0}^{\infty}{a_n}$ - сходится $ \xRightarrow{\text{по сл-вию из th. Коши}} \sum_{k=1}^{\infty}{a_n} $ - сходится
  \item Пусть $\sum_{k=1}^{\infty}{b_k}$ - расходится $\xRightarrow{\text{по сл-вию из th. Коши}} \sum_{k=N_0}^{\infty}{b_k}$ - расходится $\xRightarrow{\text{по th 3}} \sum_{k=N_0}^{\infty}{\frac{c}{2} b_n}$ - расходится $\xRightarrow{\text{по th 5}} \sum_{k=N_0}^{\infty}{a_n}$ - расходится $ \xRightarrow{\text{по сл-вию из th. Коши}} \sum_{k=1}^{\infty}{a_n} $ - расходится
 \end{enumerate}
\end{proof}
\subsubsection{Телескопический признак. Эталонный ряд $\sum_{k=1}^{\infty}{\frac{1}{n^p}}$}
\begin{theorem}[Телескопический признак]
 Пусть $a_n \searrow , a_n \geq 0$ ряд $\sum_{k=1}^{\infty}{a_n} $ - сходится $\Leftrightarrow$ сходится $\sum_{k=0}^{\infty}{2^n a_{2^n}}$
\end{theorem}
\begin{proof}
 Правый ряд $a_1 + 2a_2 + 4a_4+\dots$
 $$\text{Рассмотрим }a_2\leq a_2 \leq a_1$$
 $$2a_4\leq a_3 + a_4 \leq 2a_2 $$
 $$ 4a_8 \leq a_5 + a_6 + a_7 +a_8 \leq 4a_4 $$
 $$ \dots$$
 $$ 2^n a_{2^{n+1}} \leq \sum_{k=2^{n}+1}^{2^{n+1}}{a_k} \leq 2^n a_{2^n} $$
 Сложим выражения в левой и правой частях
 $$ a_2 + 2a_4 + 4a_8 + \dots + 2^na_{2^{n+1}} \leq S_{2^{n+1}}^A - a_1 \leq S_n^B$$
 $$\frac{1}{2} (S_{n+1}^B - a_1) \leq S_{2^{n+1}}^A - a_1 \leq S_n^B$$
 Рассмотрим отдельно каждое неравенство
 \begin{enumerate}
  \item $S_{2^{n+1}}^A - a_1 \leq S_n^B$
        Если ряд $S_n^B$ - сходится $\Rightarrow S_{2^{n+1}}^A - a_1$ - сходится $\Rightarrow S_n^A \leq S^B$ и $\{ S_n^A \}\nearrow \xRightarrow{\text{По th. Вейерштрасса}}$ ряд $\sum_{k=1}^{\infty}{a_n}$ - сходится
  \item $\frac{1}{2} (S_{n+1}^B - a_1) \leq S_{2^{n+1}}^A - a_1$
        $$S_{n+1}^B \leq 2S_{2^{n+1}}^A - a_1$$
        Если ряд $\sum_{k=1}^{\infty}{a_n}$ - сходится $\Rightarrow S_n^B \leq 2S_{2^{n+1}}^A - a_1 $ и $\{ S_n^B \}\nearrow\xRightarrow{\text{По th. Вейерштрасса}}$ ряд $\sum_{k=1}^{\infty}{2^n a_{2^n}}$ - сходится
 \end{enumerate}
 \textit{Примечание:} Расхождение доказывается по признаку сравнения\\
\end{proof}
\begin{theorem}
 Ряд $\sum_{k=1}^{\infty}{\frac{1}{n^p}}$ $\begin{cases}
   \text{cходится} ,   & \text{если} p > 1    \\
   \text{расходится} , & \text{если} p \leq 1 \\
  \end{cases}$
\end{theorem}
\begin{proof} \mbox{}\\
 \begin{itemize}
  \item Пусть $p > 1 \Rightarrow 0 < \{\frac{1}{n^p} \} \searrow_0 $
        Рассмотрим ряд из th 7 $\sum_{k=0}^{\infty}{2^n \frac{1}{2^{np}}} = \sum_{k=0}^{\infty}{(2^{1-p})^n}$ - геометрическая прогрессия $q = 2^{1-p} \Rightarrow \sum_{k=1}^{\infty}{(2^{1-p})^n}$ - сходится $\xRightarrow{\text{по th 7}} \sum_{k=1}^{\infty}{\frac{1}{n^p}}$ - сходится
  \item Пусть $p \leq 1$
        $$\frac{1}{n^p} > \frac{1}{n}\ \forall{n} \in \mathbb{N} \Rightarrow \text{ т.к } \frac{1}{n} \text{ - расходится }
         \xRightarrow{\text{по признаку сравнения}} \frac{1}{n^p} \text{ - расходится} $$
 \end{itemize}
\end{proof}
\subsubsection{Признак Коши. Признак Даламбера}
\begin{theorem}[признак Даламбера]
 Пусть $a_n > 0 $ и $\exists{\overline{\lim}_{n \rightarrow \infty}{\frac{a_{n+1}}{a_n}}} = q$, тогда
 \begin{enumerate}
  \item Если $ 0 \leq q < 1$, то ряд $\sum_{k=1}^{\infty}{a_n} $ - сходится
  \item Если $ q > 1$, то ряд $\sum_{k=1}^{\infty}{a_n} $ - расходится
  \item Если $ q = 1$, то признак не работает
 \end{enumerate}
\end{theorem}
\begin{comment}
Признак удобно применять, если в $a_n$ есть $2^n, n!, n^k$
\end{comment}
\begin{proof}
 $$\forall{\varepsilon} > 0\ \exists{N > 0}:\ \forall{n > N}\  |\frac{a_{n+1}}{a_n} - q| < \varepsilon$$
 $$- \varepsilon < \frac{a_{n+1}}{a_n} -q <\varepsilon  $$
 $$ (q-\varepsilon)a_n < \frac{a_{n+1}}{a_n} < (q+\varepsilon)a_n\ \ \ \forall{n} = N + 1, N+2,\dots$$
 \begin{enumerate}
  \item Пусть $ q < 1$ \\
        Возьмем $\varepsilon: \widetilde{q} = q + \varepsilon < 1$ (Например $\varepsilon = \frac{1-q}{2}$)
        Тогда $\forall{n} > N$
        $$ a_{n+1} < a_n \widetilde{q}$$
        $$ N+1: a_{n+2} < a_{n+1} \widetilde{q}$$
        $$ N+2: a_{n+3} < a_{n+2} \widetilde{q} <  a_{n+1} \widetilde{q}$$
        $$ \dots $$
        $$ N+k-1: a_{n+k} < a_{n+k-1} \widetilde{q} < \dots <a_{n+1} \widetilde{q}^{k-1}$$
        Т.к $\sum_{k=1}^{\infty}{\widetilde{q}^k}$ - геометрическая прогрессия ($\widetilde{q}<1$)$\Rightarrow$ сходится $\xRightarrow{\text{по признаку сравнения}} \sum_{n=1}^{\infty}{a_{n+k}}$ - сходится $\xRightarrow{\text{Следствие критерия Коши}} \sum_{n=1}^{\infty}{a_n}$ - сходится
  \item Пусть $q > 1$ \\
        Возьмем $\varepsilon: \widetilde{q} = q - \varepsilon > 1$ \\
        Тогда $\forall{n} > N\ a_n \widetilde{q} <a_{n+1} \xRightarrow{\text{по 1 пункту}} a_{n+k} >a_{n+1} \widetilde{q}^{k-1}$
        $$a_{n+1}\widetilde{q}^{k-1} \xcancel{\xrightarrow{n \rightarrow \infty}}\ \ 0 \Rightarrow a_{n+k}\xcancel{\xrightarrow{n \rightarrow \infty}}\ \ 0 \xRightarrow{\text{по необходимому признаку}} \text{ряд расходится}$$
  \item Пусть $q=1$$$\sum_{n=1}^{\infty}{\frac{1}{n}}:\ \lim_{n \rightarrow \infty}{\frac{\frac{1}{n+1}}{\frac{1}{n}}} = 1   \text{ ряд расходится} $$
        $$\sum_{n=1}^{\infty}{\frac{1}{n^2}}:\ \lim_{n \rightarrow \infty}{\frac{\frac{1}{(n+1)^2}}{\frac{1}{n^2}}} = 1  \text{ ряд сходится}$$
 \end{enumerate}
\end{proof}
\begin{exmp}
 $$\sum_{n=1}^{\infty}{\frac{n^2}{2^n}}, \lim_{n \rightarrow \infty}{\frac{\frac{(n+1)^2}{2^{n+1}}}{\frac{n^2}{2^n}}} = \lim_{n \rightarrow \infty}{\frac{(n+1)^2}{2n^2}} = \frac{1}{2} < 1 \Rightarrow \text{ряд сходится} $$
\end{exmp}
\begin{theorem}[радикальный признак Коши] \mbox{}\\
 Пусть $a_n \geq 0 $ и $\exists{\overline{\lim}_{n \rightarrow \infty}{\sqrt[n]{a_n}}} = q$, тогда
 \begin{enumerate}
  \item Если $ 0 \leq q < 1$, то ряд $\sum_{k=1}^{\infty}{a_n} $ - сходится
  \item Если $ q > 1$, то ряд $\sum_{k=1}^{\infty}{a_n} $ - расходится
  \item Если $ q = 1$, то признак не работает
 \end{enumerate}
\end{theorem}
\begin{comment}
Признак удобно применять, если в $a_n$ есть $2^n, n^k$
\end{comment}
\begin{proof}
 $$\forall{\varepsilon} > 0\ \exists{N > 0}:\ \forall{n > N}\  |\sqrt[n]{a_n}} - q| < \varepsilon$$
 $$- \varepsilon < \sqrt[n]{a_n}} -q <\varepsilon  $$
 $$ (q-\varepsilon) < \sqrt[n]{a_n}} < (q+\varepsilon)\ \ \ \forall{n} = N + 1, N+2,\dots$$
 \begin{enumerate}
  \item Пусть $q < 1 \Rightarrow \exists{\widetilde{q}} = \frac{q+1}{2} | q < \widetilde{q} < 1 \exists{N > 0}:\ \forall{n > N}\ $\\
        $$ \sqrt[n]{a_n}} < \widetilde{q} \Leftrightarrow a_n < \widetilde{q}^n\ \ \ \sum_{n=1}^{\infty}{\widetilde{q}^n} \text{ - геометрическая прогрессия} (\widetilde{q} < 1)} $$
        $$\Rightarrow \text{ ряд сходится} \xRightarrow{\text{по признаку сравнения}} \text{сходится исходный ряд }$$
  \item Пусть $q > 1 \Rightarrow \exists{\{ a_{n_k}\}}: \sqrt[n]{a_{n_k}}} > 1 \Rightarrow a_{n_k} > 1 \Rightarrow \lim_{n \rightarrow \infty}{a_{n_k}} \ne 0 \Rightarrow$ ряд расходится
  \item Пусть $ q = 1$
        $$ \sum_{n=1}^{\infty}{\frac{1}{n} : \lim_{n \rightarrow \infty}{\frac{1}{\sqrt[n]{n}}}} = 1  \text{ ряд расходится} $$
        $$ \sum_{n=1}^{\infty}{\frac{1}{n^2} : \lim_{n \rightarrow \infty}{\frac{1}{\sqrt[n]{n^2}}}} = 1  \text{ ряд сходится} $$
 \end{enumerate}
\end{proof}
\begin{exmp}
 $$\sum_{n=1}^{\infty}{(\frac{n-1}{n})^{n^2}}, \lim_{n \rightarrow \infty}{(\frac{n-1}{n})^{n}}=e^{-1} < 1 \Rightarrow \text{ряд сходится}$$
\end{exmp}
\begin{theorem}[Признак Раабе]
 Пусть $a_n > 0 $ и $\exists{\lim_{n \rightarrow \infty}{n(\frac{a_n}{a_{n+1}}-1)}} = q$, тогда
 \begin{enumerate}
  \item Если $ q < 1$, то ряд $\sum_{k=1}^{\infty}{a_n} $ - расходится
  \item Если $ q > 1$, то ряд $\sum_{k=1}^{\infty}{a_n} $ - сходится
  \item Если $ q = 1$, то признак не работает
 \end{enumerate}
\end{theorem}
\begin{exmp}
 $$\sum_{n=1}^{\infty}{\frac{1 \cdot 3 \cdot 5 \cdot \textit{\dots} \cdot (2n-1)}{2^n \cdot n!}}$$
 \textit{Доломбер: } $$  \lim_{n \rightarrow \infty}{\frac{(1 \cdot 3 \cdot 5 \cdot \textit{\dots} \cdot (2n+1)) 2^n \cdot n!} {2^{n+1} \cdot (n+1)!(1 \cdot 3 \cdot 5 \cdot \textit{\dots} \cdot (2n-1))}} =$$
 $$\lim_{n \rightarrow \infty}{\frac{2n+1}{2\cdot(n+1)}} = 1 $$
 \textit{Раабе: } $$ \lim_{n \rightarrow \infty}{n(\frac{2(n+1)}{2n+1}-1)} = \lim_{n \rightarrow \infty}{\frac{n}{2n+1}} = \frac{1}{2} < 1 \Rightarrow \text{ ряд расходится}$$
\end{exmp}

\subsubsection{Число e, как сумма ряда}

$$e = \lim_{n \rightarrow \infty}{(1+\frac{1}{n})^n} = \lim_{n \rightarrow \infty}{e_n} $$
\begin{enumerate}
 \item $$ e_n = (1 + \frac{1}{n})^n = \sum_{n=0}^{\infty}{C_k^n \frac{1}{n^k}} = \sum_{k=0}^{n}{\frac{n!}{k!(n-k)!} \frac{1}{n^k}} = $$
       $$= \sum_{k=0}^{n}{\frac{(n-k+1)(n-k+2)\dots n}{k! n^k}} < \sum_{k=0}^{n}{\frac{n^k}{k!n^k}}=\sum_{k=0}^{n}{\frac{1}{k!}} = S_n \Rightarrow e_n < S_n $$
 \item Пусть $m<n$
       $$e_n = \sum_{k=0}^{n}{C_n^k \frac{1}{n^k}} > \sum_{k=0}^{m}{C_n^k \frac{1}{n^k}}= $$
       $$ = 1 + 1 + \frac{n(n-1)}{2! n^2} + \frac{n(n-1)(n-2)}{3! n^3}+ \dots + \frac{n(n-1)(n-2)\dots(n-m+1)}{m! n^m} = A_{n,m}$$
       $$ \Rightarrow e_n > A_{n,m} \text{ Зафиксируем m. Тогда при } n\rightarrow \infty \ e \geq S_m$$
 \item $| \Rightarrow e_n < S_n \leq e \text{ и } \{S_n\} \nearrow \xRightarrow{\text{По th Вейерштрасса}} \text{Ряд сходится}$
       $$e = 1 + 1 + \frac{1}{2!} + \frac{1}{3!} + \dots + \frac{1}{n!} + \dots $$
 \item Оценка погрешности при приблежении числа e частичными суммами:
       $$ 0 < e - S_n = \sum_{k=n+1}^{\infty}{\frac{1}{k!}} = \frac{1}{(n+1)!} +\frac{1}{(n+2)!} + \frac{1}{(n+3)!} + \dots =$$
       $$= \frac{1}{(n+1)!} [ 1 + \frac{1}{(n+2)} + \frac{1}{(n+2)(n+3)}+\dots ]<$$$$< \frac{1}{(n+1)!} [ 1 + \frac{1}{(n+1)} + \frac{1}{(n+1)^2} + \frac{1}{(n+1)^3} +\dots] = \frac{1}{(n+1)!} \frac{n+1}{n} = \frac{1}{n \cdot n!}$$
\end{enumerate}
\upr Доказать, что e - иррационально
\upr Доказать, что 2<e<3
\subsection{Признак Лейбница. Признак Дирихле. Признак Абеля}
\subsubsection{}
%\setcounter{theorem}{0}
\definition Ряд вида $a_1 - a_2 + a_3 - a_4 + ... = \sum_{n=1}^{\infty}{(-1)^{n-1} a_n}$ называется знакочередующимся, где $\forall{n}\ a_n > 0$
\begin{theorem}[Признак сходимости Лейбница знакочередующихся рядов]
 Пусть дан знакочередующийся ряд $\sum_{n=1}^{\infty}{(-1)^{n-1} a_n}\ a_n > 0$\\
 Если $\lim_{n \rightarrow \infty}{a_n} = 0 $ и $\{a_n\}\searrow$, то ряд сходится
\end{theorem}
\begin{proof}
 $$S_{2m} = a_1 - a_2 + a_3 - a_4 + \dots - a_{2m-2} + a_{2m-1} - a_{2m} =$$
 $$= a_1 - (a_2 - a_3) - (a_4 - a_5) - \dots - (a_{2m-2} - a_{2m-1}) - a_{2m} \leq a_1 \Rightarrow $$
 $$ \Rightarrow \{ S_{2m}\} \text{ - ограничена сверху}$$
 $$ S_{2m+2} = S_{2m} + (a_{2m+1} - a_{2m+2}) \geq S_{2m} \Rightarrow \{S_{2m}\}\nearrow |\xRightarrow{\text{по th Вейерштрасса}} \exists{\lim_{n \rightarrow \infty}{S_{2m}}} = S $$
 $$ S_{2m+1} = S_m +a_{2m+1} \xRightarrow{S_m \rightarrow S, a_{2m+1} \rightarrow 0 \text{ При } n\rightarrow \infty} \lim_{n \rightarrow \infty}{S_{2m+1}} = S $$
\end{proof}
\comment Признак достаточный, но не необходимый!
\begin{consequence*}[Оценка для остаточного члена ряда Лейбница]
 $$ S = S_n + R_n = S_n + (-1)^n a_{n+1} + (-1)^{n+1} a_{n+2}+ \dots $$
 $$ |R_n| = |(-1)^n| |(a_{n+1} - a_{n+2}) + (a_{n+2} - a_{n+3}) + \dots | = $$ $$ = a_{n+1} - (a_{n+2} - a_{n+3}) - (a_{n+4} - a_{n+5}) - \dots \leq a_{n+1}$$
 $$ | \Rightarrow |R_n| \leq a_{n+1} $$
\end{consequence*}
\begin{lemma*}[Абеля]
 Дано $a_1, \dots , a_n $ и $b_1, \dots , b_n $. При этом:
 \begin{enumerate}
  \item $\{a_i\}$ монотонно
  \item $\exists{B} > 0 : |\sum_{i=1}^{m}{b_i}| \leq B$, где $m = 1, 2, 3, \dots, n$
 \end{enumerate}
 Тогда выполняется неравенство $|\sum_{k=1}^{n}{a_k b_k}| \leq B (|a_1|+2|a_n|)$
\end{lemma*}
\begin{proof}[]
 Обозначим $B_1 = b_1, B_2 = b_1 + b_2, \dots, B_n = b_1 + b_2 +  \dots + b_n$, также перепишем (2) условие, как $\exists{B} > 0 : |B_m| \leq B$, где $m = 1, 2, 3, \dots, n$
 $$ \sum_{k=1}^{n}{a_k b_k} = a_1 b_1 + a_2 b_2 + a_n b_n = a_1 B_1 + a_2 (B_2 - B_1) + \dots + a_n(B_n - B_{n-1}) =$$
 $$ =B_1 (a_1 - a_2) + B_2(a_2 - a_3)+ \dots + B_{n-1}(a_{n-1} - a_n) + B_n a_n $$
 $$ |\sum_{k=1}^{n}{a_k b_k}| =|B_1| |a_1 - a_2| + \dots + |B_{n-1}||a_{n-1} - a_n| + |B_n||a_n| \leq$$
 $$\leq B (\underbrace{|a_1 -a_2| + |a_2 - a_3| + \dots + |a_{n-1} - a_n|}_{\text{По монотонности, либо все модули раскроются с +, либо все с -}} + |a_n|) =$$
 $$  = B (|a_1 - a_n| + |a_n|) \stackrel{\text{Использую неравенство треугольника}}{ \leq} B(|a_1| + 2|a_n|)$$
\end{proof}
\begin{theorem}[Признак Дирихле]
 Пусть дан ряд $\sum_{n=1}^{\infty}{a_n b_n}$ тогда, если:
 \begin{enumerate}
  \item $a_n$ монотонно стремится к 0
  \item $\exists{C}\ \forall{n}\ |\sum_{k=1}^{m}{b_k}| \leq C$
 \end{enumerate}
 Ряд $\sum_{n=1}^{\infty}{a_n b_n}$ сходится
\end{theorem}
\begin{proof}
 Из (1) условия $a_n$ - монотонно $\rightarrow 0 \xRightarrow{\text{По определению предела}} \forall{\varepsilon}>0 \exists{N > 0}:\ \forall{n > N}\ |a_n| < \frac{\varepsilon}{6C}$ \\
 Из (2) условия $\forall{n}\ \forall{p}\ |\sum_{k=n}^{n+p}{b_k}| = |\sum_{k=1}^{n+p}{b_k} - \sum_{k=1}^{n-1}{b_{k}}| \leq 2C$ \\
 Рассмотрим $a_n, \dots, a_{n+p}$ и $b_n, \dots , b_{n+p} \xRightarrow{\text{По лемме Абеля ($B=2C$)}} |\sum_{k=1}^{n}{a_k b_k}| \leq 2C(|a_n| + 2|a_{n+p}|) \leq 6C \frac{\varepsilon}{6C} = \varepsilon \xRightarrow{\text{По критерию Коши}} $ ряд $\sum_{n=1}^{\infty}{a_n b_n}$ сходится
\end{proof}
\begin{theorem}[Признак Абеля]
 Пусть дан ряд $\sum_{n=1}^{\infty}{a_n b_n}$ тогда, если:
 \begin{enumerate}
  \item $a_n$ монотонно и ограничена
  \item $\sum_{k=1}^{\infty}{b_k}$ - сходится
 \end{enumerate}
 Ряд $\sum_{n=1}^{\infty}{a_n b_n}$ сходится
\end{theorem}
\begin{proof}
 Из (1) условия $\Rightarrow |a_n| \leq A \ \forall{n}$\\
 Из (2) условия $\xRightarrow{\text{По пр. Коши}} \forall{\varepsilon} > 0 \exists{N > 0}:\ \forall{n > N}\ |\sum_{k=n}^{n+p}{b_k}| < \frac{\varepsilon}{3A}$\\
 Рассмотрим $a_n, a_{n+1}, \dots, a_{n+p}$ и $b_n, b_{n+1}, \dots, b_{n+p}$
 $$ \xRightarrow{\text{По лемме Абеля. $\{a_i\}$ - монот. из усл, $B = \frac{\varepsilon}{3A}$} } |\sum_{k=n}^{n+p}{a_k b_k}| \leq B(|a_n| + 2|a_{n+p}|)=$$
 $$= \frac{\varepsilon}{3A} (\overbrace{|a_n|}^{ \leq A} + 2\overbrace{|a_{n+p}|}^{ \leq A}) \leq \varepsilon $$
 Тогда по кр. Коши для ряда $\sum_{k=1}^{\infty}{a_k b_k}$ - сходится
\end{proof}
\begin{proof}[\textit{Докозательство. Теоремы 1. Вариант 2}]
 Из условия, что $\{ a_n \} \searrow к 0$ и того факта, что $|\sum_{n=1}^{m}{(-1)^n}| \leq 1$, по теореме Дирихле следует, что ряд $\sum_{n=1}^{\infty}{(-1)^{n-1}a_n}$ - сходится
\end{proof}
\begin{exmp}
 $$ \sum_{n=1}^{\infty}{\frac{sin(nx)}{n}}\ \forall{x} \in \mathbb{R}$$
 Зафиксируем $m$. Тогда
 $$ a_n = \frac{1}{n} \searrow \text{ к } 0, b_n = sin(nx) $$
 $$ S_m = sin(x) + sin(2x) + sin(3x) + \dots + sin(mx) =$$
 $$= \frac{1}{sin(x/2)} ( sin(x)sin(x/2) + sin(2x)sin(x/2) + \dots + sin(mx)sin(x/2))  =$$
 $$= \frac{1}{2sin(x/2)} (cos(x/2) - cos(3x/2) + \dots + cos(\frac{2m-1}{2}x) - cos(\frac{2m+1}{2}x)  =$$
 $$ = \frac{1}{2sin(x/2)}(cos(x/2)- cos(\frac{2m+1}{2}x) )$$
 Если $sin(x/2) = 0 \Rightarrow sin(nx) = 0 \Rightarrow$ ряд сходится \\
 Если $sin(x/2) \ne 0 \Rightarrow |\sum_{k=1}^{n}{sin(nx)}| \leq \frac{2}{2sin(x/2)} = \frac{1}{sin(x/2)} = C$\\
 $| \xRightarrow{\text{по признаку Дирихле}}$ ряд сходится
\end{exmp}
\upr $$\sum_{n=1}^{\infty}{\frac{sin(n)cos(\frac{\pi}{4})}{2}}$$
\subsection{Абсолютно и условно сходящиеся ряды}
\begin{definition*}
 Числовой ряд $\sum_{n=1}^{\infty}{a_n}$ называется абсолютно сходящимся, если сходится ряд $\sum_{n=1}^{\infty}{|a_n|}$
\end{definition*}

\begin{theorem}
 Абсолютный ряд сходится
\end{theorem}
\begin{proof}
 Дано $\sum_{n=1}^{\infty}{|a_n|}$ - сходится. Докажем, что сходится ряд $\sum_{n=1}^{\infty}{a_n}$\\
 $\sum_{n=1}^{\infty}{|a_n|}$ - сходится $\xRightarrow{\text{по th. Коши}} \forall{\varepsilon} > 0\ \exists{N > 0}:\ \forall{n > N}\ \forall{p}\ |a_n| + |a_{n+1}| + \dots + |a_{n+p}| < \varepsilon $\\
 Рассмотрим ряд $\sum_{n=1}^{\infty}{a_n}  \xRightarrow{\text{th. Коши}} \forall{\varepsilon} > 0 \exists{N > 0}:\ \forall{n > N}\ \forall{p}\ |a_n + a_{n+1} + \dots + a_{n+p}| \leq |a_n| + |a_{n+1}| + \dots + |a_{n+p}|< \varepsilon \Rightarrow $ ряд $\sum_{n=1}^{\infty}{a_n}$ - сходится
\end{proof}
\begin{exmp}
 $ \sum_{n=1}^{\infty}{\frac{(-1)^n}{n}}$ - сходится по признаку Лейбница, но ряд $\sum_{n=1}^{\infty}{|\frac{(-1)^n}{n}|}$ - расходитя $\Rightarrow$ ряд сходится не абсолютно
\end{exmp}
\begin{definition*}
 Ряд называется условно сходящимся, если он сходится не абсолютно
\end{definition*}
\begin{theorem}[Общие свойства абсолютно сходящихся рядов]
 \begin{enumerate}
  \item Если $\sum_{n=1}^{\infty}{a_n}, \sum_{n=1}^{\infty}{b_n}$ - абсолютно сходящиеся ряды, то ряд $\sum_{n=1}^{\infty}{\lambda a_n + \mu b_k}$ абсолютно сходится
  \item Если $\overline{\lim}_{n \rightarrow \infty}{|\frac{a_{n+1}}{a_n}|} = q$ или $\overline{\lim}_{n \rightarrow \infty}{\sqrt[n]{|a_n|}} = q$, тогда ряд \begin{cases}
         \text{абсолютно сходитя}, & $q<1$ \\
         \text{расходитя},         & $q>1$ \\
         \text{не работает},       & $q=1$ \\
        \end{cases}
 \end{enumerate}
\end{theorem}
%TODO: Дописать доказательство
\begin{proof}
 \mbox{}\\
 \begin{enumerate}
  \item Критерий Коши для 1 и 2 и неравенство треугольника
  \item По аналогии с признаками для положительных рядов
 \end{enumerate}
\end{proof}
\begin{theorem}
 Пусть ряд $\sum_{n=1}^{\infty}{a_n}$ абсолютно сходится. А ряд $\sum_{n=1}^{\infty}{b_n}$ получен путем произвольной перестановки членов $a_n$\\
 Тогда ряд $\sum_{n=1}^{\infty}{b_n}$ абсолютно сходится и $\sum_{n=1}^{\infty}{b_n} = \sum_{n=1}^{\infty}{a_n}$
\end{theorem}
\begin{proof}
 $f: \mathbb{N} \xrightarrow[\text{на}]{1-1}\mathbb{N}$ В самом деле это биекция так, как\\ $\forall{k}\ \exists{i}\ b_k = a_{f(i)}$ и $\forall{n}\ \exists{i}\ a_n = b_{f^{-1}(i)}$
 \begin{enumerate}
  \item $\forall{m}\ \exists{l}: \{ b_1 , \dots , b_m \} \subset \{a_1, \dots , a_l \} \Rightarrow (l \geq m)$
  \item $\forall{l}\ \exists{m'}: \{ a_1 , \dots , a_l \} \subset \{b_1, \dots , b_{m'} \} \Rightarrow (m' \geq l)$
 \end{enumerate}
 \begin{itemize}
  \item $A = \sum_{n=1}^{\infty}{a_n}    \sum_{n=1}^{m}{|b_n|} \leq \sum_{n=1}^{l}{|a_n|}$ и $\{ S_n^B \} \searrow \xRightarrow{\text{по th Вейерштрасса}}$ ряд $\sum_{n=1}^{\infty}{b_n}$ - сходится
  \item Пусть  $\sum_{n=1}^{\infty}{a_n} = a$ и  $\sum_{n=1}^{\infty}{b_n} = b$
        Тогда:
        $$\forall{\varepsilon}\ \exists{N_1}: \sum_{n\geq N_1}{|a_n|} < \varepsilon $$
        $$\forall{\varepsilon}\ \exists{N_2}: \sum_{n\geq N_2}{|b_n|} < \varepsilon $$
        Возьмем $N = max\{ N_1, N_2 \}$\\
        Пусть $m \geq N |\sum_{n=1}^{\infty}{b_n} - b|< \varepsilon \xRightarrow{\text{Из (1) пункта }} \exists{l(m)} \xRightarrow{\text{Из (2) пункта}} \exists{m'(l)} $
        $$ \{b_1, \dots, b_m \} \subset \{a_1, \dots, a_l \} \subset \{b_1, \dots, b_{m'} \} $$
        $$ \sum_{n=1}^{l}{a_n} = \sum_{n=1}^{m}{b_n} + \underbrace{\sum_{j>m}{b_j}}_{\text{Те которые не вошли в первую сумму. Обозначис ее за $C$} } $$
        $$ |C| \leq \sum_{j>m}^{m'}{|b_j|} < \varepsilon $$
        $$ |\sum_{n=1}^{l}{a_k} - a| < \varepsilon$$
        $| \Rightarrow |a-b| = |\sum_{n=1}^{m}{b_n} - b - \sum_{n=1}^{l}{a_n} + a + c | < 3 \varepsilon \xRightarrow{\text{Т.к $\varepsilon$ произвольное}} a = b$
 \end{itemize}
\end{proof}
\begin{theorem}
 Есди ряды $\sum_{n=1}^{\infty}{a_n}, \sum_{n=1}^{\infty}{b_n}$ - абсолютно сходится, то ряд, составленный из все возможных попарных произведений $a_m b_n$, расположенных в произвольном порядке, также сходится абсолютно, и если ряд $\sum_{n=1}^{\infty}{a_n} = S^A $, а ряд $\sum_{n=1}^{\infty}{b_n} = S^B$, то сумма полученного ряда равна $S=S^A S^B$
\end{theorem}
\begin{proof}
 Расположем $a_m b_n$ в удобном порядке:
 $$ a_1 b_1, a_1 b_2, \dots, a_1 b_n, \dots $$
 $$ a_2 b_1, a_2 b_2, \dots, a_2 b_n, \dots $$
 $$ \dots $$
 $$ a_m b_1, a_m b_2, \dots, a_m b_n, \dots $$
 $$ \dots $$
 $$ a_1 b_1 + a_1 b_2 + a_2 b_2 + a_2 b_1 + \dots \eqno(1)$$
 Рассмотрим ряд:
 $$ |a_1 b_1| + |a_1 b_2| + |a_2 b_2| + |a_2 b_1| + \dots $$
 Введем обозначения:$S_n^A = \sum_{k=1}^{n}{a_k} \rightarrow S^A, S_n^B = \sum_{k=1}^{n}{b_k} \rightarrow S^B$\\
 Для того, чтобы доказать, что ряд сходится абсолютно, достаточно доказать, что существует хотябы одна подпоследовательность частичных сумм ограниченных сверху\\
 $S_n \nearrow$, т.к числа положительные
 $$S_1 = |a_1 b_1| = S_1^A \cdot S_1^B \leq S^A S^B$$
 $$S_4 = |a_1 b_1| + |a_1 b_2| + |a_2 b_2| + |a_2 b_1| = (|a_1| + |a_2|)(|b_1| + |b_2|)=S_2^A \cdot S_2^B \leq S^A S^B$$
 $$ \dots $$
 $$S_{n^2} = |a_1 b_1| + \dots + |a_1 b_n| + \dots + |a_n b_1| = (|a_1| + \dots + |a_n|)(|b_1| + \dots + |b_n|)=S_n^A \cdot S_n^B \leq S^A S^B$$
 $$| \Rightarrow S_{n^2} \leq \underbrace{S^A S^B}_{\text{Конечное число в силу абс. схожд.}}$$
 $$  \Rightarrow S_n \leq S^A S^B, S_n \nearrow \xRightarrow{\text{По th. Вейерштрасса}} \exists{\lim_{n \rightarrow \infty}{S_n}} = S $$
 $  \Rightarrow $ ряд (1) сходится абсолютно $ \xRightarrow{\text{th 3}} $ исходный ряд сходится\\
 Докажем, что $S = S^A S^B$\\
 По th3 вновь переставим ряд "удобно"
 $$ \widetilde{S_n^A} = \sum_{k=1}^{n}{a_k}, \widetilde{S_n^B} = \sum_{k=1}^{n}{b_k}$$
 $$ \underbrace{\widetilde{S_{n^2}}}_{\rightarrow \tilde{S}} = \underbrace{\widetilde{S_n^A}}_{\rightarrow \tilde{S^A}}\cdot \underbrace{\widetilde{S_n^B}}_{\rightarrow \tilde{S^B}} \Rightarrow S = S^A S^B$$
\end{proof}
\comment Пусть есть ряды $\sum_{n=1}^{\infty}{a_n}, \sum_{n=1}^{\infty}{b_n}, \sum_{n=1}^{\infty}{a_n + b_n}$, тогда ряд $\sum_{n=1}^{\infty}{a_n}$ - сходится $\Leftrightarrow$ сходятся ряды $\sum_{n=1}^{\infty}{b_n}, \sum_{n=1}^{\infty}{a_n+b_n}$
\begin{theorem}[Римана]
 Если ряд $\sum_{n=1}^{\infty}{a_n}$ сходится условно, то $\forall{A}$ можно так переставить члены ряда, что $\sum_{n=1}^{\infty}{a_n} = A$
\end{theorem}
\begin{proof}
 \mbox{}\\
 Пусть $a_1^+ , \dots , a_n^+, \dots $ - неотрицательные члены последовательности, взятые по порядку\\
 Пусть $a_1^- , \dots , a_n^-, \dots $ - отрицательные члены последовательности, взятые по порядку\\
 По условию ряд $\sum_{n=1}^{\infty}{a_n}$ - сходится, но ряд $\sum_{n=1}^{\infty}{|a_n|}$ - расходится $\Rightarrow$
 $$ a_1^+ + \dots + a_n^+ + \dots \xrightarrow{n \rightarrow \infty} +\infty$$
 $$ a_1^- + \dots + a_n^- + \dots \xrightarrow{n \rightarrow \infty} -\infty$$
 В каждой этой сумме бесконечное число слагаемых, иначе н.н.н все члены одного знака, и сходимость эквивалентна абсолютной. Если нет нитого ни другого (оба конечны), тогда ряд будет абсолютно сходится(противоречие). Если одно есть, но нет другого, тогда ряд расходится\\
 Пусть A\geq0:\\
 $  n_1 : \begin{cases}
   n_1 = 1,                                                                    & a_1^+ \geq A \\
   a_1^+ + \dots + a_{n_1-1}^+ \leq A < a_1^+ + \dots + a_{n-1}^+ + a_{n_1}^+, & a_1^+ < A    \\
  \end{cases}$
 \mbox{}\\
 $$ S_{n_1} = a_1^+ + \dots + a_{n_1}^+, |S_{n_1} - A| < a_{n_1}^+$$
 $$ n_2: a_1^+ + \dots + a_{n_1}^+ a_1^- + \dots + a_{n_2 - 1}^- \leq A < a_1^+ + \dots + a_{n_1}^+ +a_1^- + \dots + a_{n_2}^- $$
 $$ S_{n_1 + n_2} = a_1^+ + \dots + a_{n_1}^+ +a_1^- + \dots + a_{n_2}^-, |S_{n_1 + n_2} - A| < |a_{n_2}^-| $$
 $ n_3: a_1^+ + \dots + a_{n_1}^+ +a_1^- + \dots + a_{n_2}^- + a_{n_1 + 1}^+ + \dots + a_{n_3 - 1}^+ \leq$\\
 $ \leq A < a_1^+ + \dots + a_{n_1}^+ +a_1^- + \dots + a_{n_2}^- + a_{n_1 + 1}^+ + \dots + a_{n_3}^+$
 $$ |S_{n_1+n_2+n_3} - A| < a_{n_3}^+$$
 $$ n_1+n_2 < n < n_1 + n_2 +n_3, S_n \nearrow $$
 $$ |S_n - A| < max\{|a_{n_2}^-|, a_{n_3}^+ \}$$
 Применив метод математической индукции легко показать, что:\\
 Если $n_1 + n_2 + \dots +n_{2k-1} + n_{2k} < n \leq n_1 + n_2 + \dots + n_{2k-1} + n_{2k} + n_{2k+1}$, то
 $$ |S_n - A| < max\{|a_{2k}^-|, a_{2k+1}^+ \}$$
 Если $n_1 + n_2 + \dots + n_{2k-1} + n_{2k} + n_{2k+1}< n \leq n_1 + n_2 + \dots + n_{2k-1} + n_{2k} + n_{2k+1} + n_{2k+2}$, то
 $$ |S_n - A| < max\{ a_{2k+1}^+, |a_{2k+2}^-| \}$$
 Итак, сформирован новый ряд.\\
 Докажем, что $S_n \xrightarrow{n \rightarrow \infty} A$\\
 Пусть $\forall{\varepsilon} > 0 $ Из сходимости ряда $\sum_{n=1}^{\infty}{a_n}$ $\xRightarrow{\text{Необходимы признак сход-ти}} \lim_{n \rightarrow \infty}{a_n} = 0 \Rightarrow \exists{N > 0}:\ \forall{n > N}\ |a_n| < \varepsilon$
 $$ | \Rightarrow |S_n - A| < |a_n| < \varepsilon \Rightarrow S_n \xrightarrow{n \rightarrow \infty} A$$
\end{proof}
\begin{upr}
 Покажите, что если ряд $\sum_{n=1}^{\infty}{a_n}$ - условно сходится, то его члены можно переставить так, чтобы он стал расходится
\end{upr}
\begin{upr}
 Докажите, что если при любой перестановки его членов ряд остаётся сходящимся, то ряд сходится абсолютно
\end{upr}

\section{Интеграл}
\subsection{Неопределенный интеграл}
\subsubsection{Первообразные}
\begin{definition*}
 Функция $F$ называется первообразной функции $f$ на некотором интервале, если $F'(x) = f(x)$
\end{definition*}

\begin{definition*}
 Функция $F: I \rightarrow \mathbb{R}$ (точная) первообразная ф-ции $f: I \rightarrow \mathbb{R}$, если $\forall{x} \in I \ F'(x) = f(x)$
\end{definition*}
\begin{memories}
 $\textit{Теорема Лагранджа о среднем.}$\\
 $ \text{Пусть} f:[a,b]\rightarrow\mathbb{R}$ - непрерывная на $[a,b]$  и дифференцируема на $(a,b)$. Тогда $\exists{c}\in (a,b): f(b)-f(a)=f'(c)(b-a)$
\end{memories}

\begin{lemma}[ о точных первообразных]
 Пусть $f: [a,b] \rightarrow \mathbb{R}$, $F_1, F_2$ - первообразные на $[a,b]$, тогда $F_1(x)=F_2(x)+C$, где $C \in \mathbb{R}$
\end{lemma}
\begin{proof}
 Рассмотрим $G(x) = F_1(x) - F_2(x)$
 $$ G'(x) = F_1'(x) - F_2'(x) = 0 $$
 $$ \forall{x,y} \in [a,b] G(x) - G(y)=G`(\Xi)(x-y) = 0$$
 Тогда $$G(x)=G(a) = C$$
\end{proof}
\begin{lemma}[из будущего]
 У всякой непрерывной на промежутке функции есть точная первообразная на этом промежутке
\end{lemma}
\begin{exmp}
 $|x| = \begin{cases}
   x,  & x\geq 0 \\
   -x, & x<0     \\
  \end{cases} \Rightarrow
  (\frac{x|x|}{2})'=|x|
 $
\end{exmp}
\begin{definition*}
 Функция $F: I \rightarrow \mathbb{R}$ называется обобщенной первообразной ф-ции $f: I \rightarrow \mathbb{R}$, если $F'(x) = f(x)$ всюду за исключением конечного числа точек и $F$ - непрерывная
\end{definition*}
\begin{exmp}
 $|x|$ - обобщенная первообразная $sign(x)$, т.к $|x|'=sign(x)$
\end{exmp}
\begin{lemma}
 Если функция кусочно непрерывна на промежутке и ограничена, тогда всегда у нее есть обобщенная первообразная на этом промежутке
\end{lemma}
\begin{lemma}[об обобщенной первообразной]
 Если $F_1, F_2: I \rightarrow \mathbb{R}$ обобщенные первообразные функции $f: I \rightarrow \mathbb{R}$, то $F_1(x)= F_2(x)+C$
\end{lemma}
\begin{proof}
 Пусть $x_1<x_2< \dots < x_n$ - точки, в которых нет $F_1'(x)$ или нет $F_2'(x)$\\
 На любом отрезке  $[x_i; x_{i+1}]$ по предыдущей лемме  $F_1$ и $F_2$ - точные первообразные ф-ции $f$ (по лемме 3)\\
 $$ F_1(x)=F_2(x) + C_i &\text{ На }[x_i; x_{i+1}]$$
 $$ F_1(x)=F_2(x) + C_{i-1} &\text{ На }[x_{i-1}; x_{i}]$$
 $$|\Rightarrow F_1(x_i)= F_2(x_i) + C_i = F_2(x_i) + C_{i-1} \Rightarrow C_i = C_{i-1}$$
\end{proof}
\begin{name}
 Неопределенным интегралом(антипроизводной) ф-ции $f$ называется символ $$\int f(x) dx$$ который обозначает либо некоторую первообразную, либо совокупность первообразных
\end{name}
\begin{name}
 $f(x) dx$ -подынтегральное выражение
\end{name}
\begin{name}
 Иногда удобно ввести обозначение: $F'(x)dx = dF(x)$
\end{name}
\begin{comment}
У $dx$ приоритет больше, чем у сложения
$$ \int (f(x) + g(x))dx \ne \int f(x) + g(x) dx$$
Но меньше, чем умножение
$$ \int f(x)g(x)dx $$
\end{comment}
\subsubsection{Приемы отыскания первообразных}
\begin{enumerate}

 \item Таблица первообразных\\
       \includegraphics{pervoobraz}
       \begin{comment}
       $$\int \frac{dx}{x} = \begin{cases}
         ln(x)+C_1,  & x>0 \\
         -ln(x)+C_2, & x<0 \\
        \end{cases}$$
       \end{comment}
       $$\int sh(x)dx = ch(x)+C_1, \int ch(x)dx = sh(x)+C_2$$

 \item линейность
       $$ \int (f(x) + g(x))dx = \int f(x)dx + \int g(x)dx$$
       $$ \int \alpha f(x) dx = \alpha \int f(x)dx$$
       \begin{exmp}
        $$ \int (x+1)^2 dx = \int (x^2 + 2x + 1)dx = \frac{x^3}{3} + x^2 + x + C $$
       \end{exmp}
       \begin{exmp}
        $$ \int \frac{x^2}{1+x^2} dx = \int \frac{x^2 + 1 -1}{1+x^2} = \int (1 - \frac{1}{1+x^2})dx = x - arctg(X) + C $$
       \end{exmp}
 \item интегрирование по частям
       \begin{comment}
       $$(f(x)g(x))'= f'(x)g(x) + f(x)g'(x)$$
       $$ \int (f'(x)g(x) + f(x)g'(x)) dx = f(x)g(x) + C $$
       $$ \int f(x)g'(x) = f(x)g(x)- \int f'(x)g(x) + C$$
       \end{comment}
       \begin{comment}
       Когда использовать: $ \int x^n f(x) dx, \int \dots ln \dots dx, \int \dots arctg \dots dx $
       \end{comment}
       \begin{exmp}
        $$ \int x sin(x) dx = \int x (-cosx)'dx = x(-cosx) - \int (-cosx)dx =$$
        $$= x(-cosx) + \int cos(x)dx=-xcosx + sinx + C$$
       \end{exmp}
       \begin{exmp}
        $$ \int lnx dx = \int x' ln(x) dx = x ln(x) - \int 1 dx = x ln(x) -x + C$$
       \end{exmp}
 \item Замена переменной
       $$f(\phi(x))'= f'(\phi(x))\phi'(x)$$
       Если $\phi(x)$ - дифференцируемая функция, то $y = \phi(x)$
       $$ \int f(\phi(y))dy = \int f(\phi(x))\phi'(x)dx$$
       \begin{exmp}
        $$ \int \frac{x}{x^2+1} dx = \int \frac{\frac{1}{2}}{x^2+1} 2x dx = \int \frac{\frac{1}{2}}{x^2+1} (x^2)' dx$$
        Обозначим $y = x^2 + 1$
        $$ y = x^2 + 1  $$
        $$ \int  \frac{\frac{1}{2}}{y}dy = \frac{1}{2}ln(y) + C = \frac{1}{2}ln(x^2 + 1) + C$$
       \end{exmp}
 \item "Стыковка"
       $$f:[a,c] \Rightarrow \mathbb{R}, a<b<c$$

       $$ F_1(x) \text{- Первообразна на }[a,b]$$
       $$ F_2(x) \text{- Первообразна на }[b,c]$$
       Тогда:
       $$ \int f(x) dx = \begin{cases}
         F_1(x) + C,               & x\in [a,b] \\
         F_2(x) + F_2(b)-F_1(b)+C, & x\in [b,c] \\
        \end{cases}$$
\end{enumerate}

\end{document}
