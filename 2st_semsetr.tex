\documentclass[a4paper]{article}
\usepackage[utf8]{inputenc}
\usepackage[english, russian]{babel}
\usepackage{indentfirst}
\usepackage{amsmath,amsfonts,amssymb,amsthm,mathtools}
\usepackage{svg}
\usepackage{graphicx}
\usepackage{prftree}
\usepackage{multicol}
\usepackage[12pt]{extsizes}
\newtheorem{theorem}{Теорема}
\newtheorem*{theorem*}{Теорема}
\newtheorem{lemma}{Лемма}
\newtheorem*{lemma*}{Лемма}
\theoremstyle{definition}
\newtheorem{definition}{Определение}
\newtheorem*{definition*}{Определение}
\newtheorem*{name}{Обозначение}
\newtheorem*{exmp}{Пример}
\newtheorem*{paradoks}{Парадокс}
\newtheorem*{hypo}{Гипотеза}
\newtheorem{proposition}{Предложение}
\newtheorem*{proposition*}{Предложение}
\newtheorem*{comment*}{Замечание}
\newtheorem{comment}{Замечание}
\newtheorem*{upr}{Упражнение}
\newtheorem{upr*}{Упражнение}
\newtheorem{consequence}{Следствие}
\newtheorem*{consequence*}{Следствие}
\newtheorem{property}{Свойство}
%\setuphead[subsubsection][style=\rm\bfa]
\newtheorem*{property*}{Свойство}

\title{ МАТАН 2 Семестр}
\author{Носорев Константин}
\date{2019\\}
\begin{document}
\maketitle
\tableofcontents
\section{ГЛАВА}
\subsection{Определение ряда. Основные свойства}
\setcounter{subsubsection}{-1}
\subsubsection{Конечные суммы}
$$a_1+a_2+\dots+a_n = \sum_{k=1}^{n} a_k$$
\begin{itemize}
 \item $\sum_{k=1}^{n}{a_n} + \sum_{k=1}^{n}{b_n} = \sum_{k=1}^{n}{(a_n+b_n)}$
 \item $\lambda \sum_{k=1}^{n}{a_n} = \sum_{k=1}^{n}{(\lambda a_n)}$
\end{itemize}
\subsubsection{Числовые ряды}
\definition{Числовым рядом называется выражение вида $$\sum_{k=1}^{\infty}{a_k} = a_1 + a_2 + \dots + a_n+ \dots $$ где $a_k \in \mathbb{R}$ - общий член последовательности, \\а $S_1 = a_1,  S_2 = a_1 + a_2, S_n = \sum_{k=1}^{n}{a_k}$ - частичные суммы ряда }
\begin{definition}
 Числовой ряд называется \textit{сходящимся}, если сходится последовательность его частичных сумм\\
 $$\lim_{n\rightarrow \infty}{S_n}=S \text{ - сумма ряда}, \sum_{k=1}^{\infty}{a_k} = S \in \mathbb{R} $$
 Если предел бесконечен или не существует, то ряд \textit{расходится}
\end{definition}
\exmp $\sum_{k=0}^{\infty}{q^k} \text{ - геометрическая прогрессия }$\\
$$S_0 = 1, S_1 = 1+q, S_2= 1 + q +q^2, \dots , S_n = \sum_{k=1}^{n}{q^k} = \frac{1-q^{n+1}}{1-q}$$\\
Если $|q| < 1$, то $S_n \xrightarrow{n\rightarrow \infty} \frac{1}{1-q}$ - ряд сходится\\
Если $|q| > 1$, то $S_n \xrightarrow{n\rightarrow \infty} \infty$ - ряд расходится\\
Если $q = 1$, то $S_n \xrightarrow{n\rightarrow \infty} \infty$ - ряд расходится\\
Если $q = -1$, то $S_n = \begin{cases}
  0, & \text{n=2k}   \\
  1, & \text{n=2k+1} \\
 \end{cases}$ - ряд расходится\\
\subsubsection{Основные  свойства}
\begin{theorem}[Критерий Коши]
 $$\sum_{k=1}^{\infty}{a_k} \Leftrightarrow\ \forall{\varepsilon}> 0\ \exists{N}:\ \forall{m\geq n > N}\  |\sum_{k=n}^{m}{a_k}|<\varepsilon$$
\end{theorem}
\begin{proof}
 Используя критерий Коши для посл-ти частичных сумм
 $$\sum_{k=1}^{\infty}{a_k}\text{ - сходится }\Leftrightarrow S_n = \sum_{k=1}^{n}{a_k} \text{ - сходится }$$ $$\xLeftrightarrow{\text{По кр. Коши}} \forall{\varepsilon}>0 \exists{N}\ :\forall{m\geq n-1 > N}\  |S_m-S_{n-1}|<\varepsilon$$
 $$\forall{\varepsilon}>0 \exists{N}\ :\forall{m\geq n \geq N+1}\ |\sum_{k=n}^{m}{a_k}|<\varepsilon$$
\end{proof}
\exmp $$\sum_{k=1}^{\infty}{\frac{1}{n}} \text{ - расходится }  \sum_{k=1}^{\infty}{\frac{1}{n^2}} \text{ - сходится } $$
\begin{consequence*}
 Если в ряду изменить произвольным образом конечное число слагаемых, то новый ряд сходится, когда сходится исходный, и новый ряд расходится, если исходный расходится
\end{consequence*}
\comment Сходимость ряда независит от поведения конечного числа слагаемых
\begin{theorem}[Необходимый признак сходимости ряда]
 Если $\sum_{k=1}^{\infty}{a_k}$ - сходится, то $\lim_{n\rightarrow\infty}{a_n}=0$
\end{theorem}
\begin{consequence*}
 Если $\lim_{n\rightarrow\infty}{a_n}\ne0$, то ряд расходится
\end{consequence*}
\begin{proof}
 $$a_n = S_n - S_{n-1} \Rightarrow \lim_{n\rightarrow\infty}{a_n} = \lim_{n\rightarrow\infty}{S_n - S_{n-1}} = 0$$
\end{proof}
\begin{theorem}[Арифметические свойства]
 Пусть ряды $\sum_{k=1}^{\infty}{a_k}\ \sum_{k=1}^{\infty}{b_k}$ - сходятся, тогда
 $$\forall{\lambda,\mu} \in \mathbb{R}\ \sum_{k=1}^{\infty}{(\lambda a_k + \mu b_k)} = \lambda \sum_{k=1}^{\infty}{a_k} + \mu \sum_{k=1}^{\infty}{b_k} \text{ - сходится}$$
\end{theorem}
\begin{proof}
 Пусть $S_n^A = \sum_{k=1}^{n}{a_k} \rightarrow S^A$
 $S_n^B = \sum_{k=1}^{n}{b_k} \rightarrow S^B$\\
 Рассмотрим $\sum_{k=1}^{\infty}{(\lambda a_k + \mu b_k)}$,$$ S_n=\sum_{k=1}^{n}{\lambda a_k + \mu b_k} = \lambda \sum_{k=1}^{n}{a_k} + \mu \sum_{k=1}^{n}{b_k} = \lambda S_n^A + \mu S_n^B\Rightarrow$$
 $$\exists{\lim_{n\rightarrow\infty}{S_n}} = \lim_{n\rightarrow\infty}{(\lambda S_n^a + \mu S_n^B)} =\lambda S^A + \mu S^B$$
\end{proof}
\begin{comment*}
 В частности $\sum_{k=1}^{\infty}{\lambda a_k} = \lambda \sum_{k=1}^{\infty}{a_k}$
\end{comment*}
\subsubsection{Неотрицательные числовые ряды}
Рассмотрим $\sum_{k=1}^{\infty}{a_k}, a_k \geq 0, S_n \nearrow$
\begin{theorem}[Критерий сходимости ряда с неотрицательными числами]
 Ряд, члены короторого неотрицательны, сходится $\Leftrightarrow$ посл-ть частичных сумм ограничена
\end{theorem}
\begin{proof}
 \mbox{}\\
 $\Rightarrow$ Ряд сходится $\xRightarrow{\text{По определению}}$ последовательность частичных сумм сходится $\xRightarrow{\text{По свойству сходящейся посл-ти}} \{S_n\}  $ - ограничена \\
 $\Leftarrow$   $\{S_n\}  $ - ограничена и $S_n \nearrow \xRightarrow{\text{По th. Вейерштрасса}}  \{S_n\} $ - сходится $\xRightarrow{\text{По определению}}$ ряд сходится
\end{proof}
\begin{theorem}[Признак сравнения]
 Пусть $$\exists{N > 0}: \forall{n > N}  \sum_{k=n}^{\infty}{a_k}, \sum_{k=n}^{\infty}{b_k}, a_k \geq 0, b_k \geq 0 \text{ и } a_k \leq b_k $$
 \begin{enumerate}
  \item Из сходимости $\sum_{k=1}^{\infty}{b_k} \Rightarrow$ сходимость ряда $\sum_{k=1}^{\infty}{a_k}$
  \item Из расходимости $\sum_{k=1}^{\infty}{a_k} \Rightarrow$ расходимость ряда $\sum_{k=1}^{\infty}{b_k}$
 \end{enumerate}
\end{theorem}
\begin{proof}
 Конченое число членов ряда не влияет на сходимость $\Rightarrow$ будем считать, что $a_k \leq b_k \forall{k \geq 1}$
 \begin{enumerate}
  \item Пусть $S_n^A=\sum_{k=1}^{n}{a_k}, S_n^B=\sum_{k=1}^{n}{b_k} $ \\
        $$\forall{k \geq 1}\ a_k \leq b_k\ \Rightarrow S_n^A \leq S_n^B\ \forall{n} \text{ , если сходится } \sum_{k=1}^{\infty}{b_k}$$
        $$\text{то }S_n^B \nearrow \text{ и сходится к }S^B \text{ при } n \rightarrow \infty \Rightarrow S_n^A \leq S_n^B \leq S^B \Rightarrow S_n^A \nearrow$$
        $$\text{ ограничена сверху } S^B \xRightarrow{\text{по th 4}} \text{ ряд } \sum_{k=1}^{\infty}{a_k} \text{ сходится }$$
  \item (от противного)Пусть $\sum_{k=1}^{\infty}{a_k}$ - расходится, а $\sum_{k=1}^{\infty}{b_k}$ - сходится, тогда по пункту 1 ряд $\sum_{k=1}^{\infty}{a_k}$ - сходится $\Rightarrow \bot$
 \end{enumerate}
\end{proof}
\begin{theorem}[Признак сравнения в предельной форме]
 Пусть $$\sum_{k=1}^{\infty}{a_k},\ \sum_{k=1}^{\infty}{b_k}, a_k \geq 0\ b_k > 0\ \text{ и }\exists{\lim_{n \rightarrow \infty}{\frac{a_n}{b_n}} = c > 0} \text{ - конечное} $$
 тогда ряды сходятся и расходятся одновременно
\end{theorem}
\begin{proof}
 $$\forall{\varepsilon > 0} \exists{N > 0}:\ \forall{n > N}\ |\frac{a_n}{b_n} - c| < \varepsilon $$
 $$-\varepsilon < \frac{a_n}{b_n} - c < \varepsilon$$
 $$-\varepsilon + c< \frac{a_n}{b_n}  < \varepsilon + c$$
 $$(-\varepsilon + c)b_n < a_n  < (\varepsilon + c)b_n$$
 Возьмем $\varepsilon = \frac{c}{2}$
 $$\exists{N_0 > 0}:\ \forall{n > N_0}\ $$
 $$ \frac{c}{2}b_n < a_n < \frac{3c}{2} b_n$$
 \begin{enumerate}
  \item Пусть $\sum_{k=1}^{\infty}{b_k}$ - сходится $\xRightarrow{\text{по сл-вию из th. Коши}} \sum_{k=N_0}^{\infty}{b_k}$ - сходится $\xRightarrow{\text{по th 3}} \sum_{k=N_0}^{\infty}{\frac{3c}{2} b_n}$ - сходится $\xRightarrow{\text{по th 5}} \sum_{k=N_0}^{\infty}{a_n}$ - сходится $ \xRightarrow{\text{по сл-вию из th. Коши}} \sum_{k=1}^{\infty}{a_n} $ - сходится
  \item Пусть $\sum_{k=1}^{\infty}{b_k}$ - расходится $\xRightarrow{\text{по сл-вию из th. Коши}} \sum_{k=N_0}^{\infty}{b_k}$ - расходится $\xRightarrow{\text{по th 3}} \sum_{k=N_0}^{\infty}{\frac{c}{2} b_n}$ - расходится $\xRightarrow{\text{по th 5}} \sum_{k=N_0}^{\infty}{a_n}$ - расходится $ \xRightarrow{\text{по сл-вию из th. Коши}} \sum_{k=1}^{\infty}{a_n} $ - расходится
 \end{enumerate}
\end{proof}
\subsubsection{Телескопический признак. Эталонный ряд $\sum_{k=1}^{\infty}{\frac{1}{n^p}}$}
\begin{theorem}[Телескопический признак]
 Пусть $a_n \searrow , a_n \geq 0$ ряд $\sum_{k=1}^{\infty}{a_n} $ - сходится $\Leftrightarrow$ сходится $\sum_{k=0}^{\infty}{2^n a_{2^n}}$
\end{theorem}
\begin{proof}
 Правый ряд $a_1 + 2a_2 + 4a_4+\dots$
 $$\text{Рассмотрим }a_2\leq a_2 \leq a_1$$
 $$2a_4\leq a_3 + a_4 \leq 2a_2 $$
 $$ 4a_8 \leq a_5 + a_6 + a_7 +a_8 \leq 4a_4 $$
 $$ \dots$$
 $$ 2^n a_{2^{n+1}} \leq \sum_{k=2^{n}+1}^{2^{n+1}}{a_k} \leq 2^n a_{2^n} $$
\end{proof}
\end{document}
